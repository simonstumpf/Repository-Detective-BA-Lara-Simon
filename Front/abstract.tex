% !TEX root = ../main.tex

%----------------------------------------------------------------------------------------
% ABSTRACT PAGE
%----------------------------------------------------------------------------------------
\begin{abstract}
\addchaptertocentry{\abstractnames} % Add the abstract to the table of contents

Code reviews are an essential part of the software development process, as they not only ensure the quality of the code, but also promote teamwork. For this reason, 
software development students should learn modern development methods such as code reviews. However, there is often a lack of transparency about the quality of collaboration and the state of the development process.
It is unclear how effectively students learn these practices if there is no knowledge of how they apply them. In addition, lecturers lack the appropriate tools to analyse students' work and identify potential problems in the teams or inefficient working patterns.

Automated analyses of repositories should help to identify and address these difficulties at an early stage. To this end, a tool was developed at the Zurich University of Applied Sciences (ZHAW) that analyses repositories and displays the results visually. In this thesis, further performance metrics are to be determined and analysis algorithms developed.
To this end, six research questions are analysed that deal with the relationship between pull request latency and churn (number of line changes), the reasons for closing pull requests, the relationship between project progress and review duration, students' working patterns, differences between part-time and full-time students and the comparison of student projects with professional open source repositories. These research questions are analysed using data from 71 student GitHub repositories and three open source projects.

The results show that there is no significant correlation between latency and churn. Similarly, the reasons for closing the pull requests cannot be clearly classified. However, there is a clear correlation between the project time and the review duration. The latter decreases steadily over the course of the project. In addition, there was a clear pattern in the students' working days. For part-time students in particular, these are concentrated on the teaching days. There were no significant differences in churn and number of pull requests between full-time and part-time students. However, part-time students show a greater variation in latency with many very short processing times. In both models, most latencies are under 30 minutes.
 A comparison of project modules and open source projects (OSS) shows similar correlations between churn, number of commits and changed files. However, there are differences in the relationship between description length and latency. While a positive correlation is recognisable in the student projects, this was partly negative or non-existent in OSS projects.


\end{abstract}


%----------------------------------------------------------------------------------------
% German ABSTRACT PAGE
%----------------------------------------------------------------------------------------
\begin{extraAbstract}
\addchaptertocentry{\extraabstractname} % Add the abstract to the table of contents
Code-Reviews sind ein wesentlicher Bestandteil des Softwareentwicklungsprozesses, da sie nicht nur die Qualität des Codes sicherstellen, sondern auch die Teamarbeit fördern. Deshalb sollen 
Studierende der Informatik moderne Entwicklungsmethoden wie Code-Reviews erlernen.  Oftmals fehlt jedoch die Transparenz über die Qualität der Zusammenarbeit sowie den Zustand des Entwicklungsprozesses.
Es ist unklar, wie effektiv Studierende diese Praktiken lernen, wenn keine Kenntnisse darüber vorliegen, wie diese angewendet werden. Zudem fehlen den Dozierenden die entsprechenden Werkzeuge, um die Arbeit der Studierenden zu analysieren und potenzielle Probleme in den Teams oder ineffiziente Arbeitsmuster zu identifizieren.

Mithilfe automatisierter Auswertungen von Repositories sollen die genannten \linebreak Schwierigkeiten frühzeitig erkannt und adressiert werden. Zu diesem Zweck wurde an der Zürcher Hochschule für Angewandte Wissenschaften (ZHAW) ein Tool entwickelt, das Repositories auswertet und die Ergebnisse visuell darstellt. In dieser Arbeit sollen weitere Performancemetriken ermittelt und Analysealgorithmen entwickelt werden. Dafür werden sechs Forschungsfragen untersucht, die sich mit dem Zusammenhang von Pull-Request-Latency und Churn (Anzahl Zeilenänderungen), den Gründen für das Schliessen von Pull-Requests, dem Zusammenhang zwischen Projektverlauf und Review-Dauer, Arbeitsmustern von Studierenden, Unterschieden zwischen Teilzeit- und Vollzeitstudierenden sowie dem Vergleich von studentischen Projekten mit professionellen Open-Source-Repositories befassen. Diese Forschungsfragen werden anhand der Daten von 71 GitHub-Repositories von Studierenden und drei Open-Source-Projekten analysiert.

 Die Ergebnisse zeigen, dass kein signifikanter Zusammenhang zwischen Latency und Churn besteht. Ebenso lassen sich die Schliessgründe der Pull-Requests nicht eindeutig klassifizieren. Es besteht jedoch ein klarer Zusammenhang zwischen der Projektzeit und der Review-Dauer. Letztere nimmt im Verlauf des Projekts stetig ab. Zusätzlich zeigte sich ein klares Muster bei den Arbeitstagen der Studierenden. Vor allem bei den Teilzeitstudierenden konzentrieren sich diese auf die Unterrichtstage. Zwischen Vollzeit- und Teilzeitstudierenden zeigen sich keine signifikanten Unterschiede bei Churn und Anzahl der Pull-Requests. Jedoch bei der Latency weisen Teilzeitstudierende eine grössere Streuung mit vielen sehr kurzen Bearbeitungszeiten auf. Bei beiden Modellen liegen die meisten Latencies unter 30 Minuten.
 Beim Vergleich von Projektmodulen und Open-Source-Projekten (OSS) zeigen sich ähnliche Korrelationen zwischen Churn, Anzahl Commits und Changed Files. Es bestehen jedoch Unterschiede im Zusammenhang zwischen Description Length und Latency. Während bei den studentischen Projekten ein positiver Zusammenhang erkennbar ist, war dieser bei OSS-Projekten teils negativ oder nicht vorhanden.
 

\end{extraAbstract}
