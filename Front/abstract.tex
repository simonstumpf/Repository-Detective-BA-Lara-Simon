% !TEX root = ../main.tex

%----------------------------------------------------------------------------------------
% ABSTRACT PAGE
%----------------------------------------------------------------------------------------
\begin{abstract}
\addchaptertocentry{\abstractnames} % Add the abstract to the table of contents
This thesis analyses the use of code reviews and pull requests in student projects at the Zurich University of Applied Sciences (ZHAW). The aim is to determine performance metrics and develop analysis algorithms. The work is based on a tool developed at the ZHAW called ‘GitGauge’. This application mines and analyses repositories and displays the results visually.

The thesis examines six research questions that deal with the relationship between pull request latency and churn (number of line edits), the reasons for closing PRs, the relationship between project progress and review duration, student work patterns, differences between part-time and full-time students, and the comparison of student projects with professional open source repositories.

 To answer these questions, data from 71 student GitHub repositories was analysed. The methodology includes intensive literature research, the iterative development of the research questions and the procedure for answering these questions. These include GraphQL queries to mine the data, analysing the data and statistical methods such as Spearman correlation. In addition, the GitGauge application was expanded.

 The results show that there is no significant correlation between latency and churn. Similarly, the reasons for closing pull requests cannot be clearly categorised. Furthermore, a clear correlation was found between the project time and the review duration. The latter decreases steadily over the course of the project. In addition, a clear pattern emerged in the students' working days. Particularly in the case of part-time students, these are concentrated on the teaching days on which the project modules take place. There are no significant differences between the two teaching models in terms of churn and the number of pull requests. In terms of latency, it can be seen that part-time students have a greater spread in the values and many have very short processing times. However, both models have most latencies under 30 minutes.
 When comparing the correlations of the metrics for the project modules and the analysed open source projects (OSS), there are similarities in the correlations between churn, number of commits and changed files. However, there are differences in the correlation between description length and latency. While there is a positive correlation in the student projects, a negative correlation was found in one of the OSS projects and no clear correlation was recognisable in the other projects.


\end{abstract}


%----------------------------------------------------------------------------------------
% German ABSTRACT PAGE
%----------------------------------------------------------------------------------------
\begin{extraAbstract}
\addchaptertocentry{\extraabstractname} % Add the abstract to the table of contents
In dieser Arbeit wird die Nutzung von Code-Reviews bzw. Pull-Requests (PR) in studentischen Projekten der Zürcher Hochschule für Angewandte Wissenschaften (ZHAW) untersucht. Das Ziel besteht darin, Performancemetriken zu ermitteln und Analysealgorithmen zu entwickeln.
Die Arbeit baut auf einem an der ZHAW entwickelten Tool namens „GitGauge” auf. Diese Applikation betreibt Mining und Analyse von Repositories und stellt die Ergebnisse visuell dar.

 In der Arbeit werden sechs Forschungsfragen untersucht, die sich mit dem Zusammenhang von Pull-Request-Latency und Churn (Anzahl Zeilenänderungen), den Gründen für das Schliessen von PRs, dem Zusammenhang zwischen Projektverlauf und Review-Dauer, Arbeitsmustern von Studierenden, Unterschieden zwischen Teilzeit- und Vollzeitstudierenden sowie dem Vergleich von studentischen Projekten mit professionellen Open-Source-Repositories befassen.

 Um diese Fragen zu beantworten, wurden Daten von 71 GitHub-Repositories von Studierenden analysiert. Die Methodik umfasst intensive Literaturrecherche, die iterative Erarbeitung der Forschungsfragen und der Vorgehensweise zur Beantwortung dieser Fragen. Diese umfassen GraphQL-Abfragen, um die Daten zu minen, die Analyse der Daten sowie statistische Verfahren wie die Spearman-Korrelation. Zusätzlich wurde die Applikation GitGauge erweitert.
 

 Die Ergebnisse zeigen, dass kein signifikanter Zusammenhang zwischen Latency und Churn besteht. Ebenso lassen sich die Schliessgründe der Pull-Requests nicht eindeutig klassifizieren. Des Weiteren wurde ein klarer Zusammenhang zwischen der Projektzeit und der Review-Dauer festgestellt. Letztere nimmt im Verlauf des Projekts stetig ab. Zusätzlich zeigte sich ein klares Muster bei den Arbeitstagen der Studierenden. Vor allem bei den Teilzeitstudierenden konzentrieren sich diese auf die Unterrichtstage, an denen die Projektmodule stattfinden. Zwischen den beiden Unterrichtsmodellen gibt es beim Churn und der Anzahl von Pull-Requests keine signifikanten Unterschiede. Bei der Latency ist erkennbar, dass Teilzeitstudierende eine grössere Streuung in den Werten haben und viele sehr kurze Bearbeitungszeiten aufweisen. Jedoch haben beide Modelle die meisten Latencies unter 30 Minuten.
 Beim Vergleich der Korrelationen der Metriken bei den Projektmodulen und den untersuchten Open-Source-Projekten (OSS) zeigen sich Übereinstimmungen bei den Zusammenhängen von Churn, Anzahl Commits und Changed-Files. Jedoch beim Zusammenhang zwischen Description-Length und Latency zeigen sich Unterschiede. Während bei den studentischen Projekten ein positiver Zusammenhang besteht, wurde bei einem der OSS-Projekte ein negativer Zusammenhang festgestellt und bei den anderen Projekten war kein klarer Zusammenhang erkennbar. 
 

\end{extraAbstract}
