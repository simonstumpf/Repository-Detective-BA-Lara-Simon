% Indicate the main file. Must go at the beginning of the file.
% !TEX root = ../main.tex

%----------------------------------------------------------------------------------------
% CHAPTER TEMPLATE
%----------------------------------------------------------------------------------------


\chapter{Diskussion und Ausblick} % Main chapter title

\label{Chapter5} % Change X to a consecutive number; for referencing this chapter elsewhere, use \ref{ChapterX}

%----------------------------------------------------------------------------------------
% SECTION 1
%----------------------------------------------------------------------------------------

\section{Diskussion}
Ziel dieser Arbeit war es, Produktivitätsmetriken aus Git-Repositories zu gewinnen und Analysealgorithmen zu definieren.  Zu diesem Zweck wurden im Rahmen einer Literaturrecherche aussagekräftige Metriken im Kontext von Code-Reviews ermittelt. Auf Basis dieser Metriken wurden sechs Forschungsfragen formuliert, die potenzielle Zusammenhänge und Einflussfaktoren im Softwareentwicklungsprozess untersuchen. Zur Beantwortung der Forschungsfragen wurden die entsprechenden Metriken mithilfe der erweiterten Applikation GitGauge aus den Repositories extrahiert und anschliessend analysiert und ausgewertet.

In diesem Kapitel werden die Forschungsfragen erneut aufgegriffen, eingeordnet und im Kontext der gewonnenen Erkenntnisse diskutiert.



\subsection{Forschungsfrage 1: Zusammenhang Latency und Churn}
Wie im Abschnitt \secref{sec:PullRequestDauer} erwähnt, besteht in der Fachliteratur keine Einigkeit über den Zusammenhang zwischen den beiden Metriken Churn und Latency \parencite{yu_wait_2015}\parencite{hasan_understanding_2023}\parencite{kudrjavets_small_2022}.
Die Ergebnisse (\secref{sec:ResultatChurnLatency}) dieser Arbeit zeigen keinen Zusammenhang, weder mit noch ohne Ausreisser. Der Spearman-Koeffizient fällt hierfür zu gering aus. 
Zusätzlich besteht auch kein Zusammenhang zwischen Latency und Churn in den Korrelationsanalysen von den überprüften OSS-Projekten in Kapitel \secref{sec:Korrelationsanalyse}. Dies zeigt, dass die nicht zusammenhängenden Metriken kein ausschliessliches Phänomen von den Racetrack-Repositories sind.

Anhand unserer Ergebnisse wird deutlich, dass Pull-Requests ein komplexes Konstrukt sind und sich die Latencies der Pull-Requests aus vielen verschiedenen Gegebenheiten und somit Metriken zusammensetzen.

\subsection{Forschungsfrage 2: Schliessgründe der PRs}
Die Analyse der Schliessgründe bezieht sich ausschliesslich auf die Projektmodule. Da die Untersuchung der Gründe manuell erfolgte und dies mit viel Aufwand verbunden war, lagen die Open-Source-Projekte ausserhalb des Scopes.

Die Untersuchung der Schliessgründe von Pull-Requests (\secref{sec:UntersuchungSchliessgründePRs}) zeigt, dass besonders häufig die Kategorie OG (ohne erkennbaren Grund) auftritt, vor allem bei kleineren PRs mit einem Churn unter 100. Dies deutet darauf hin, dass kleinere Änderungen tendenziell weniger dokumentiert und begründet geschlossen werden. 
Bei den  Teilzeitklassen tritt dieses Verhalten deutlich häufiger auf. Eine mögliche Erklärung dafür ist Zeitmangel oder eine geringere Relevanz kleinerer Änderungen im Vergleich zu grösseren / wichtigeren Beiträgen. Ebenfalls zeigen die Analysen in \secref{sec:ErgebnisseEntwicklungsaktivtät}, dass die Pull-Requests oftmals an den Projektmodul-Unterrichtstagen geschlossen werden. Deshalb könnte es sein, dass die Reviews in Person und daher ohne Dokumentation durchgeführt werden. 

Bei PRs mit einem grossem Churn zeigen sich in den Vollzeitklassen häufiger strukturierte Schliessgründe wie ZFB (falscher Zielbranch) oder IA (Implementierung
abgelehnt). Bei den Teilzeitklassen sind es FPI (Feature durch anderen
PR implementiert) und DIV (Divers). Bei beiden ist jedoch die Kategorie OG (ohne erkennbaren Grund) am häufigsten vertreten. 
Die hohe Anzahl an PRs ohne erkennbaren Grund legt nahe, dass die Kommunikation über andere Kanäle (z.B. Microsoft Teams, Jira, WhatsApp) oder direkt vor Ort erfolgt. In solchen Fällen werden Pull-Requests möglicherweise mündlich besprochen und im Anschluss geschlossen, ohne dass ein schriftlicher Kommentar im PR selbst hinterlassen wird.

Die normalisierten Daten zeigen, dass in Vollzeitklassen durchschnittlich mehr PRs pro Projekt geschlossen werden. Dies kann darauf hindeuten, dass der Review-Prozess dort konsequenter durchgeführt wird.

Ein weiterer Unterschied zeigt sich im zeitlichen Verlauf der Schliessungen. In Teilzeitklassen treten viele geschlossene PRs erst in den letzten Tagen des Projekts auf, besonders am Tag der Abgabe. In den Vollzeitklassen sind die Schliessungen gleichmässiger über den gesamten Projektverlauf verteilt.
Diese Häufung zum Projekt\-ende in Teilzeitklassen könnte mehrere Ursachen besitzen: Zum einen steht Teilzeitstudierenden durch ihre beruflichen Verpflichtungen während des Semesters oftmals nur ein begrenztes Zeitfenster für die Projektarbeit zur Verfügung. Meistens bündelt sich dies auf die Unterrichtstage oder Abende. Des Weiteren ist die synchrone Abstimmung innerhalb der Teams durch die eingeschränkte Verfügbarkeit der Mitglieder erschwert, was zu einer Ansammlung offener PRs führen kann. Ebenfalls scheint der Fokus in Teilzeitklassen eher auf der termingerechten Abgabe als auf einem optimierten Entwicklungsprozess zu liegen, was zur Folge hat, dass viele PRs erst kurz vor Projektende finalisiert werden.

Zusammenfassend lässt sich feststellen, dass sich die Schliessgründe von Pull-Re\-quests nicht zuverlässig klassifizieren lassen. Viele Pull-Requests werden ohne erkennbaren Grund geschlossen. Bei den klassifizierbaren PRs zeigen sich Unterschiede sowohl in der Verteilung der Gründe als auch in der zeitlichen Struktur. Vollzeitklassen schliessen PRs tendenziell strukturierter und über den gesamten Projektverlauf verteilt, während Teilzeitklassen häufiger spontane oder unkommentierte Schliessungen am Projektende aufweisen.


\subsection{Forschungsfrage 3: Zusammenhang Projektverlauf und Review-\linebreak Dauer}
Die Ergebnisse der \fref{forschungsfrage3} zeigen deutlich, dass die latency im Verlauf der Projekte abnimmt, obwohl gleichzeitig die Churn-Werte steigen. Besonders an den letzten Tagen vor Projektende wurden viele Pull-Requests sehr schnell geschlossen, teilweise in einer Geschwindigkeit, die keine regulären und genauen Reviews mehr zulässt. 

Dabei ist zu beachten, dass diese Projekte, wie bereits in Kapitel \secref{sec:Projektmodule} erwähnt, einen festen Abgabetermin haben und eine verspätete Abgabe Auswirkungen auf die Bewertung hat. Eine naheliegende Erklärung ist, dass Studierende in den letzten Tagen vor Abgabe versuchen, möglichst viele Funktionalitäten zu implementieren und bestehende Fehler zu beheben. Daraus ergibt sich eine grosse Anzahl an Pull-Requests, die unter Zeitdruck schnell gemerged werden. 

Dieser Zeitdruck kann zur Folge haben, dass Reviews nur oberflächlich oder gar nicht durchgeführt werden. Jedoch könnte es auch sein, dass die Entwicklung im Pair-Programming (zwei Entwickler programmieren zusammen auf einem Notebook) stattfindet. Diese Methodik wurde von unserem Projektteam in gewissen Fällen in den jeweiligen Modulen angewendet. Da der Code gemeinsam erstellt wird, kann ein separates Review entfallen, da die Überprüfung direkt bei der Implementierung erfolgt.


\subsection{Forschungsfrage 4: Patterns der Zusammenarbeit hinsichtlich der Arbeitstage}
Die Analyse der Arbeitstage von Pull-Requests (\secref{sec:ErgebnisseEntwicklungsaktivtät}) zeigt deutliche Unterschiede der Arbeitstage zwischen Vollzeit- und Teilzeitklassen. Bei der Betrachtung der Tage, an denen PRs erstellt und bearbeitet wurden, fällt auf, dass sich die Aktivitäten in den Teilzeitklassen stark auf spezifische Wochentage konzentrieren, insbesondere auf die Unterrichtstage. 

Die Auswertung der Erstellungsdaten zeigt, dass in Teilzeitklassen ein Grossteil der PRs an den Unterrichtstagen sowie an Wochenenden erstellt wird. Eine mögliche Erklärung ist, dass Teilzeitstudierende unter der Woche aufgrund anderer Verpflichtungen im beruflichen Umfeld nur eingeschränkt Zeit für Projektarbeit finden.

Bei den Vollzeitklassen zeigt sich hingegen eine gleichmässigere Verteilung der Pull-Request-Aktivitäten über die Woche. Diese Gruppen arbeiten typischerweise an allen Wochentagen ausser am Wochenende, was auf eine kontinuierlichere Arbeitsweise hinweist. Der entsprechende Unterrichtstag der Klasse sticht hier besonders heraus, da an diesem Tag vergleichsweise viele PRs erstellt und bearbeitet wurden. Eine mögliche Erklärung dafür könnte sein, dass an diesem Tag oftmals Gruppensitzungen oder gemeinsame Entwicklungstreffen (Weeklys) stattfinden. 

Diese unterschiedlichen Arbeitsrhythmen der beiden Unterrichtsmodelle bedeuten, dass Dozierende auf verschiedene Rahmenbedingungen treffen. Diese sollten bei Betreuung und Bewertung der Projekte berücksichtigt werden.

\subsection{Forschungsfrage 5: Unterschiede Voll- und Teilzeitklassen hinsichtlich Nutzung der PRs}
In der \fref{forschungsfrage5} wurde untersucht, ob Unterschiede in der Nutzung von Pull-Requests zwischen den Vollzeit- und Teilzeitklassen bestehen, insbesondere im Hinblick auf Anzahl, Umfang (Churn) und Bearbeitungsdauer (Latency).

Im Median zeigen die Unterschiede bei den Churns keine signifikanten Unterschiede auf. Betrachtet man jedoch den Mittelwert, so liegt dieser bei den Vollzeitstudierenden um rund 40 Zeilen höher. Zudem ist die Standardabweichung bei den Vollzeitklassen deutlich grösser (+286), was auf eine grössere Streuung der Änderungsumfänge hindeutet. Auch der maximale Churn-Wert ist in dieser Gruppe deutlich höher. 

Eine detaillierte Analyse zeigt, dass beide Unterrichtsmodelle den grössten Anteil ihrer Pull-Requests mit Änderungen im Bereich von 50–199 Zeilen einreichen. Sehr grosse Pull-Requests (über 10'000 Zeilen) sind in beiden Gruppen selten, treten jedoch bei den Vollzeitstudierenden etwas häufiger auf. Dies erklärt die grössere Streuung. Bemerkenswert ist jedoch, dass 40\,\% dieser sehr grossen PRs von den Vollzeitstudierenden wieder geschlossen wurden, was darauf hindeutet, dass solche umfangreichen Änderungen nicht automatisch akzeptiert werden.


Hinsichtlich der Latency ergibt sich ein unterschiedliches Bild: Der Median ist bei den Teilzeitstudierenden geringer, was auf eine insgesamt schnellere durchschnittliche Bearbeitung hindeutet. Gleichzeitig sind jedoch sowohl der Mittelwert als auch die Standardabweichung höher, verursacht durch einige sehr lange Pull-Requests. Die Extremwerte zeigen, dass Teilzeitstudierende sowohl sehr kurze PRs (unter einer Minute) als auch sehr lange (über sieben Tage) häufiger haben als Vollzeitstudierende.

Die meisten Pull-Requests in beiden Gruppen werden innerhalb von 30 Minuten bearbeitet. Die längeren Bearbeitungszeiten bei den Teilzeitstudierenden lassen sich plausibel durch das in \fref{forschungsfrage4} aufgezeigte Arbeitsverhalten erklären: Da die Projektarbeit hauptsächlich an bestimmten Unterrichtstagen erfolgt, bleiben PRs unter der Woche häufiger liegen.

Im Durchschnitt erstellen Vollzeitstudierende acht Pull-Requests mehr pro Projekt als die Teilzeitstudierenden. Ein klarer Grund für diesen Unterschied lässt sich aus den übrigen Daten nicht ableiten, insbesondere da weder kleinere Churns noch kürzere Latencies bei den Vollzeitgruppen dominieren. 

\subsection[Forschungsfrage 6: Unterschiede der Repository-Metriken zwischen Studentenprojekten und professionellen GitHub-Organi\linebreak sationen]{Forschungsfrage 6: Unterschiede der Repository-Metriken zwischen Studentenprojekten und professionellen GitHub-Organisationen}
Die für die \fref{forschungsfrage6} benötigte Analyse der Repository-Metriken zwischen den Studentenprojekten (Racetrack-Repositories) und den professionellen GitHub-Organisationen Ubique, Schweizerische Bundesbahn (SBB) und Zalando zeigt spezifische Unterschiede, welche anhand mehrerer Korrelations-Heatmaps visualisiert wurden. 

Im Vergleich mit Ubique zeigen die Racetrack-Repositories stärkere positive Korrelationen zwischen Description Length und Commits sowie Description Length und Churn. Dies deutet darauf hin, dass in den Studentenprojekten lange Beschreibungen häufiger mit grossen Änderungen und mehreren Commits verbunden sind. Dies kann als Hinweis dienen, dass bei den Projektmodul-Repositories verstärkt auf das Erklären der Änderungen und auf das Reflektieren im Review-Prozess geachtet wird, möglicherweise als Teil der Bewertungskriterien. 

Der Vergleich mit der Schweizerischen Bundesbahnen (SBB) zeigt, dass die Race\-track-Projekte eine stärkere positive Korrelation zwischen Description Length und Commits sowie Churn haben. Zudem besitzen die Racetrack-Projekte eine etwas schwächere Korrelation zwischen Comments und den anderen Metriken im Vergleich zur SBB.

Im Gegensatz dazu zeigt sich bei Zalando eine negative Korrelation zwischen Description Length und Churn, was bedeutet, dass grössere Änderungen tendenziell weniger ausführlich dokumentiert werden. Dies könnte auf optimierte, informellere oder stärker automatisierte Prozesse in professionellen Organisationen hindeuten. Ebenfalls muss berücksichtigt werden, dass bei grösseren Open-Source-Projekten, von welchen Zalando mehrere verfügt (z.B. Postgres Kubernetes Operator, Skipper HTTP Router oder Logbook), diverse PRs aus der Community stammen und dementsprechend weniger standardisiert oder formal beschrieben sind \parencite{noauthor_zalandologbook_2025}\parencite{noauthor_zalandoskipper_2025}\parencite{noauthor_zalandopostgres-operator_2025}.

Zusammenfassend lassen sich die wesentlichen Unterschiede zwischen den untersuchten Studentenprojekten und professionellen Organisationen vor allem in der Art und Weise der Beschreibung und Dokumentation von Pull-Requests zeigen. Die Racetrack-Projekte verwenden tendenziell längere und detailliertere Beschreibungen bei grösseren Änderungen, während in den untersuchten professionellen Organisationen diese Zusammenhänge weniger deutlich oder sogar negativ ausgeprägt sind.


\section{Ausblick}
Die vorliegenden Analysen liefern diverse Einblicke in das Pull-Request-Verhalten der Studierenden, insbesondere in den Projektmodul-Repositories. 
Auf Basis der gewonnenen Erkenntnisse ergeben sich verschiedene Punkte zur weiterführenden Forschung sowie der konkreten Erweiterung des GitGauge-Tools. 

\subsection{Erweiterung von GitGauge}
Die im Rahmen dieser Arbeit untersuchten Metriken könnten teilweise direkt in GitGauge-Auswertungen aufgenommen werden. Besonders interessant wären:
\begin{itemize}
\item \textbf{Automatisierte Klassifikation von Schliessgründen}: Die Kategorisierung der Pull-Request-Schliessgründe erfolgte in dieser Arbeit manuell. Eine automatisierte Erkennung auf Basis der Kommentaranalyse mittels eines LLMs könnte implementiert werden. Dies könnte dazu führen, dass die Gründe besser klassifiziert werden können. Dadurch stünde Dozierenden eine zusätzliche Metrik zur Beurteilung der Teamkommunikation und des Review-Prozesses zur Verfügung.
\item \textbf{Integration von CI/CD-Metriken}: Derzeit werden CI/CD-Metriken in GitGauge nicht berücksichtigt, da CI/CD in den Projektmodulen PM1, PM2 und PM3 nicht Teil des Lehrplans ist. Für eine erweiterte Nutzung des Tools, insbesondere für das Projektmodul 4 oder im professionellen Umfeld, wäre die Integration von Metriken wie Build-Zeit, Testdurchläufe oder Anzahl fehlgeschlagener Pipelines eine wertvolle Erweiterung.
\item \textbf{Analyse von Teamdynamiken}: Ergänzend zu den bestehenden Metriken \linebreak könnten neue Visualisierungen zur Zusammenarbeit innerhalb der Teams implementiert werden. Beispielsweise könnten Netzwerkdiagramme auf Basis von Kommentaren aufzeigen, welche Teammitglieder miteinander interagieren. Ein Aktivitätsindex pro Person würde darüber hinaus helfen, individuelle Beiträge transparenter darzustellen. Solche Visualisierungen könnten zudem verdeutlichen, ob es innerhalb eines Teams isolierte Untergruppen (Inseln) gibt oder ob eine gleichmässige Zusammenarbeit stattfindet.
\end{itemize}

\subsection{Weiterführende Forschung}
Aus den in dieser Arbeit durchgeführten Analysen ergeben sich weiterführende Fragestellungen, die über die aktuellen Projektmodule hinausgehen:
\begin{itemize}
\item \textbf{Nachhaltigkeit und weitere Metriken durch CI/CD}: Insbesondere für das Projektmodul PM4 oder für externe Industrieprojekte könnte untersucht werden, wie sich CI/CD auf Qualität und Reviewprozesse auswirkt. So könnte ein Fokus auf nachhaltige Praktiken gelegt werden. Beispielsweise könnten Tests selektiv ausgeführt werden (z.B. nur beim Öffnen eines Pull-Requests anstatt bei jedem Commit), um Ressourcen zu schonen und die Effizienz zu steigern.
\item \textbf{Vergleich mit Enterprise-Projekten}: Eine Erweiterung des Vergleichs zwischen den Studierendenprojekten und professionellen Organisationen könnte weitere Unterschiede in Kultur, Formalisierung und Tooling aufdecken. Besonders spannend wären Analysen von grossen Organisationen, welche keine öffentliche GitHub-Organisation besitzen.
\item \textbf{Einfluss von Teamstruktur und Berufserfahrung}: Die aktuell untersuchten Unterschiede zwischen Vollzeit- und Teilzeitstudierenden könnten noch detaillierter hinsichtlich Berufserfahrung, Vorwissen oder Teamkonstellation \linebreak analysiert werden. Eine solche Analyse würde jedoch den Rahmen dieser Arbeit überschreiten, da die dafür erforderlichen Daten nicht allgemein zugänglich sind und eine eigene Erhebung, beispielsweise in Form von Umfragen, notwendig wäre.
\end{itemize}


\subsection{Fazit}

Die Ergebnisse dieser Arbeit zeigen, dass Pull-Requests mehr als nur ein technisches Tool im Entwicklungsprozess sind. Sie spiegeln auch die Dynamik von Teamprozessen, Zeitmanagement und Kommunikationsverhalten wider. Durch gezieltes Repository-Mining lassen sich daraus wertvolle Erkenntnisse für die Lehre, Projektsteuerung und Prozessverbesserung gewinnen.

Die analysierten Metriken liefern dabei konkrete Ansatzpunkte, um Qualität, Zusammenarbeit und Nachhaltigkeit zu messen.
Die vorliegenden Resultate bilden eine solide Basis für weiterführende Forschungsarbeiten, etwa zur Wirkung von CI/CD-Praktiken, zur Analyse von Teamstrukturen oder zum Vergleich mit industriellen Projekten.