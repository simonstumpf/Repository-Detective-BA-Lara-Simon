% Indicate the main file. Must go at the beginning of the file.
% !TEX root = ../main.tex

%----------------------------------------------------------------------------------------
% CHAPTER TEMPLATE
%----------------------------------------------------------------------------------------


\chapter{Diskussion und Ausblick} % Main chapter title

\label{Chapter5} % Change X to a consecutive number; for referencing this chapter elsewhere, use \ref{ChapterX}

%----------------------------------------------------------------------------------------
% SECTION 1
%----------------------------------------------------------------------------------------

\section{Diskussion}

Wie in der Aufgabenstellung beschrieben, wurden im Rahmen einer Literaturrecherche relevante Metriken identifiziert und aus den Repositories extrahiert. Auf dieser Grundlage wurden sechs Forschungsfragen definiert, die sich auf spezifische Metriken beziehen und deren Zusammenhänge untersuchen sollen. In diesem Kapitel werden die Forschungsfragen erneut aufgegriffen, eingeordnet und im Kontext der gewonnenen Erkenntnisse diskutiert.


\subsection{Diskussion Forschungsfrage 1: Zusammenhang Latency und Churn}
Wie im Abschnitt \secref{sec:PullRequestDauer} erwähnt, besteht in der Fachliteratur keine Einigkeit über den Zusammenhang zwischen den beiden Metriken \textit{Churn} und \textit{Latency} \parencite{yu_wait_2015}\parencite{hasan_understanding_2023}\parencite{kudrjavets_small_2022}. \\
Die Ergebnisse (\secref{sec:ResultatChurnLatency}) dieser Forschungsfrage zeigen keinen Zusammenhang, weder mit noch ohne Ausreisser. Der Spearman-Koeffizient fällt hierfür zu gering aus. \\
Berücksichtigt man zusätzlich den Kontext aus \fref{forschungsfrage3}, wird deutlich, dass die Projektdauer Einfluss auf die \textit{Latency} hat. Gegen Ende eines Projekts verkürzt sich die \textit{Latency}, während der \textit{Churn} zunimmt. Dieser gegenläufige Verlauf schwächt einen direkten Zusammenhang zwischen \textit{Churn} und \textit{Latency} ab. \\
Die Korrelationsanalyse der Racetrack-Repositories zeigt zudem, dass andere Metriken, wie etwa die Anzahl der Kommentare, Einfluss auf die \textit{Latency} haben, jedoch nicht auf dem \textit{Churn}. Da Kommentare auf Änderungsbedarf hinweisen können, und dies unabhängig von der \textit{Churn}-Metrik, erklärt  ebenfalls, warum zwischen \textit{Latency} und \textit{Churn} kein klarer Zusammenhang besteht. \\
Zusätzlich besteht auch kein Zusammenhang zwischen \textit{Latency} und \textit{Churn} in den Korrelationsanalysen von den überprüften OSS-Projekten in Kapitel \secref{sec:Korrelationsanalyse}. Dies zeigt, dass die nicht zusammenhängenden Metriken kein ausschliessliches Phänomen von den Racetrack-Repositories sind.

Anhand unserer Ergebnisse wird deutlich, dass Pull Requests ein komplexes Konstrukt sind und sich die \textit{Latencies} der Pull Requests aus vielen verschiedenen Gegebenheiten und somit Metriken zusammensetzen.

\subsection{Diskussion Forschungsfrage 2: Schliessgründe der PRs}
Die Untersuchung der Schliessgründe von Pull Requests (\secref{sec:UntersuchungSchliessgründePRs}) zeigt, dass besonders häufig die Kategorie \textit{OG} (ohne erkennbaren Grund) auftritt, vor allem bei kleineren PRs mit einem Churn unter 100. Dies deutet darauf hin, dass kleinere Änderungen tendenziell weniger dokumentiert und begründet geschlossen werden. \\
Bei den  Teilzeitklassen tritt dieses Verhalten deutlich häufiger auf. Eine mögliche Erklärung dafür ist Zeitmangel oder eine geringere Relevanz kleinerer Änderungen im Vergleich zu grösseren / wichtigeren Beiträgen. Ebenfalls zeigen die Analysen in \secref{sec:ErgebnisseEntwicklungsaktivtät}, dass die Pull Requests oftmals an den Projektmodul-Unterrichtstagen geschlossen werden. Deshalb könnte es sein, dass die Reviews in Person und daher ohne Dokumentation durchgeführt werden. 

Bei PRs mit einem grossem Churn zeigen sich in den Vollzeitklassen häufiger strukturierte Schliessgründe wie \textit{ZFB} (falscher Zielbranch) oder \textit{IA} (Implementierung
abgelehnt). Bei den Teilzeitklassen sind es \textit{FPI} (Feature durch anderen
PR implementiert) und \textit{DIV} (Divers). Bei beiden ist jedoch die Kategorie \textit{OG} (ohne erkennbaren Grund) am häufigsten vertreten. \\
Die hohe Anzahl an PRs ohne erkennbaren Grund legt nahe, dass die Kommunikation über andere Kanäle (z.B. Microsoft Teams, Jira, WhatsApp) oder direkt vor Ort erfolgt. In solchen Fällen werden Pull Requests möglicherweise mündlich besprochen und im Anschluss geschlossen, ohne dass ein schriftlicher Kommentar im PR selbst hinterlassen wird.

Die normalisierten Daten zeigen, dass in Vollzeitklassen durchschnittlich mehr PRs pro Projekt geschlossen werden. Dies kann darauf hindeuten, dass der Review-Prozess dort konsequenter durchgeführt wird.

Ein weiterer Unterschied zeigt sich im zeitlichen Verlauf der Schliessungen. In Teilzeitklassen treten viele geschlossene PRs erst in den letzten Tagen des Projekts auf, besonders am Tag der Abgabe. In den Vollzeitklassen sind die Schliessungen gleichmässiger über den gesamten Projektverlauf verteilt.  \\
Diese Häufung zum Projektende in Teilzeitklassen könnte mehrere Ursachen besitzen: Zum einen stehen Teilzeitstudierenden durch ihre berufliche Verpflichtungen während dem Semester oftmals nur ein begrenztes Zeitfenster für die Projektarbeit zur Verfügung. Meistens bündelt sich dies auf die Unterrichtstage oder Abende. Des weiteren ist die synchrone Abstimmung innerhalb der Teams durch die eingeschränkte Verfügbarkeit der Mitglieder erschwert, was zu einer Ansammlung offener PRs führen kann. Ebenfalls scheint der Fokus in Teilzeitklassen eher auf der termingerechten Abgabe als auf einem optimierten Entwicklungsprozess zu liegen, was zur Folge hat, dass viele PRs erst kurz vor Projektende finalisiert werden.

Zusammenfassend lässt sich feststellen, dass sich die Schliessgründe von Pull Requests nicht zuverlässig klassifizieren lassen. Viele Pull Requests werden ohne erkennbaren Grund geschlossen. Bei den klassifizierbaren PRs zeigen sich Unterschiede sowohl in der Verteilung der Gründe als auch in der zeitlichen Struktur. Vollzeitklassen schliessen PRs tendenziell strukturierter und über den gesamten Projektverlauf verteilt, während Teilzeitklassen häufiger spontane oder unkommentierte Schliessungen am Projektende aufweisen.


\subsection{Diskussion Forschungsfrage 3: Zusammenhang Projektverlauf auf Review-Dauer}
Die Ergebnisse der \fref{forschungsfrage3} zeigen deutlich, dass die \textit{latency} im Verlauf der Projekte abnimmt, obwohl gleichzeitig die \textit{Churn}-Werte steigen. Besonders an den letzten Tagen vor Projektende wurden viele Pull Requests sehr schnell geschlossen, teilweise in einer Geschwindigkeit, die keine regulären und genauen Reviews mehr zulässt. 

Dabei ist zu beachten, dass diese Projekte, wie bereits in Kapitel \secref{sec:Projektmodule} erwähnt, einen festen Abgabetermin haben und eine verspätete Abgabe Auswirkungen auf die Bewertung hat. Eine naheliegende Erklärung ist, dass Studierende in den letzten Tagen vor Abgabe versuchen, möglichst viele Funktionalitäten zu implementieren und bestehende Fehler zu beheben. Daraus ergibt sich eine grosse Anzahl an Pull Requests, die unter Zeitdruck schnell gemergt werden. \\
Dieser Zeitdruck kann zur Folge haben, dass Reviews nur oberflächlich oder gar nicht durchgeführt wurden. In manchen Fällen erfolgt die Entwicklung möglicherweise im Pair-Programming (zwei Entwickler programmieren zusammen auf einem Notebook), wodurch ein separates Review entfällt, da der Code gemeinsam erstellt wurde.

\subsection{Diskussion Forschungsfrage 4: Patterns der Zusammenarbeit hinsichtlich der Arbeitstage}
Die Analyse der Arbeitstage von Pull Requests (\secref{sec:ErgebnisseEntwicklungsaktivtät}) zeigt deutliche Unterschiede der Arbeitstage zwischen Vollzeit- und Teilzeitklassen. Bei der Betrachtung der Tage, an denen PRs erstellt und bearbeitet wurden, fällt auf, dass sich die Aktivitäten in den Teilzeitklassen stark auf spezifische Wochentage konzentrieren, insbesondere auf die Unterrichtstage. 

Die Auswertung der Erstellungsdaten zeigt, dass in Teilzeitklassen ein Grossteil der PRs an den Unterrichtstagen sowie an Wochenenden erstellt wird. Eine mögliche Erklärung ist, dass Teilzeitstudierende unter der Woche aufgrund anderer Verpflichtungen im privaten oder beruflichen Umfeld nur eingeschränkt Zeit für Projektarbeit finden.

Bei den Vollzeitklassen zeigt sich hingegen eine gleichmässigere Verteilung der Pull Request-Aktivitäten über die Woche. Diese Gruppen arbeiten typischerweise an allen Wochentagen ausser am Wochenende, was auf eine kontinuierlichere Arbeitsweise hinweist. Der Mittwoch respektive der entsprechende Unterrichtstag der Klasse sticht hier besonders heraus, da an diesem Tag vergleichsweise viele PRs erstellt und bearbeitet wurden. Eine mögliche Erklärung dafür könnte sein, dass an diesem Tag oftmals Gruppensitzungen oder gemeinsame Entwicklungstreffen (\textit{Weeklys}) stattfinden. \\
Diese unterschiedlichen Arbeitsrhythmen der beiden Unterrichtsmodelle bedeuten, dass Dozierende auf verschiedene Rahmenbedingungen treffen. Diese sollten bei Betreuung und Bewertung der Projekte berücksichtigt werden.

\subsection{Diskussion Forschungsfrage 5: Unterschiede Voll- und Teilzeitklassen hinsichtlich Nutzung PR}
In der \fref{forschungsfrage5} wurde untersucht, ob es Unterschiede in der Nutzung der Pull Requests hinsichtlich Anzahl, Umfang und Dauer zwischen Vollzeit- und Teilzeitklassen gibt.

Im Median zeigen die Unterschiede bei den Churns keine signifikanten Unterschiede auf. Jedoch beträgt der Unterschied beim Mittelwert 40 Zeilen. Bei der Standardabweichung sieht man, dass die Vollzeitstudierenden einen um 286 grösseren Wert haben. Ebenso haben die Vollzeitstudierenden einen deutlich höheren Maximalwert bei den Churns. Bei der genaueren Analyse zeigte sich, dass beide Unterrichtsmodelle die meisten Pull Requests mit einer Änderung von 50–199 Zeilen erstellen und die wenigsten mit einer Churn-Size von über 10 000 Zeilenänderungen. Die Vollzeitstudierenden haben jedoch mehr von solchen grossen Churns als die Teilzeitstudierenden, was eine Erklärung für die grössere Standardabweichung sein kann. Jedoch werden bei den Vollzeitstudierenden 40\,\% von diesen Pull Requests geschlossen. Somit zeigt sich, dass es im Allgemeinen keinen signifikanten Unterschied zwischen den Unterrichtsmodellen gibt. Die Verteilung der Pull Requests über die Churns ist in beiden Klassen sehr ähnlich. 

Bei der Latency ist der Median bei den Teilzeitstudierenden kleiner, der Durchschnitt und die Standardabweichung jedoch grösser. Dies lässt sich durch den Maximalwert erklären. Dieser ist bei den Teilzeitstudierenden deutlich höher. Bei der genaueren Analyse zeigte sich, dass Teilzeitstudierende im Gegensatz zu den Vollzeitstudierenden sehr kurze PRs (unter einer Minute) und sehr lange PRs (über sieben Tage) haben, wobei die Vollzeiklassen eine ausgewogenere Verteilung haben. Beide Unterrichtsmodelle weisen jedoch die höchste Anzahl von Latencies im Bereich unter 30 Minuten auf. Dies zeigt, dass viele Pull Requests relativ zeitnah bearbeitet werden. Die längeren Latencies könnten mit den Ergebnissen aus \fref{forschungsfrage4} zusammenhängen. Dort kam heraus, dass die Studierenden vor allem an den Projektmodultagen gearbeitet haben. Eine plausible Erklärung wäre somit, dass die Studierenden sich fixe Tage für dieses Projekt eingeplant haben und die Pull Requests daher liegen bleiben.

In der Anzahl unterscheiden sich die Teilzeit- von den Vollzeitklassen. Die Vollzeitstudierenden erstellen im Projekt im Durchschnitt acht Pull Requests mehr als Teilzeitklassen. Da Vollzeitstudierende jedoch keine besonders kleinen Pull Requests erstellen und auch keine geringeren Latencies aufweisen, ist kein spezieller Grund dafür ersichtlich.

\subsection{Diskussion Forschungsfrage 6: Unterschiede der Repository Metriken zwischen Studentenprojekten und professionellen Github Organisationen}
Die für die \fref{forschungsfrage6} benötigte Analyse der Repository-Metriken zwischen den Studentenprojekten (Racetrack Repositories) und den professionellen Github-Organisationen Ubique, Schweizerische Bundesbahn (SBB) und Zalando zeigt spezifische Unterschiede, welche anhand mehrerer Korrelations-Heatmaps visualisiert wurden. 

Im Vergleich mit Ubique zeigen die Racetrack Repos stärkere positive Korrelationen zwischen \textit{Description Length} und \textit{Commits} sowie \textit{Description Length} und \textit{Churn}. Dies deutet darauf hin, dass in den Studentenprojekten lange Beschreibungen häufiger mit grossen Änderungen und mehreren Commits verbunden sind. Dies kann als Hinweis sein, dass bei den Projektmodul-Repositories verstärkt auf das Erklären der Änderungen und auf das Reflektieren im Review-Prozess geachtet wird, möglicherweise als Teil der Bewertungskriterien. 

Der Vergleich mit der Schweizerischen Bundesbahnen (SBB) zeigt, dass die Racetrack-Projekte eine stärkere positive Korrelation zwischen \textit{Description Length} und \textit{Commits} sowie \textit{Churn} haben. Zudem besitzen die Racetrack-Projekte eine etwas schwächere Korrelation zwischen \textit{Comments} und den anderen Metriken im Vergleich zur SBB.

Im Gegensatz dazu zeigt sich bei Zalando eine negative Korrelation zwischen \textit{Description Length} und \textit{Churn}, was bedeutet, dass grössere Änderungen tendenziell weniger ausführlich dokumentiert werden. Dies könnte auf optimierte, informellere oder stärker automatisierte Prozesse in professionellen Organisationen hindeuten. Ebenfalls muss berücksichtigt werden, dass bei grösseren OpenSource-Projekten, über welche Zalando mehrere verfügt (z.B. Postgres Kubernetes Operator, Skipper HTTP Router oder Logbook), diverse PRs aus der Community stammen und dementsprechend weniger standardisiert oder formal beschreiben sind \parencite{noauthor_zalandologbook_2025} \parencite{noauthor_zalandoskipper_2025} \parencite{noauthor_zalandopostgres-operator_2025}.

Zusammenfassend lassen sich die wesentlichen Unterschiede zwischen den untersuchten Studentenprojekten und professionellen Organisationen vor allem in der Art und Weise der Beschreibung und Dokumentation von Pull Requests zeigen. Die Racetrack-Projekte verwenden tendenziell längere und detailliertere Beschreibungen bei grösseren Änderungen, während in den untersuchten professionellen Organisationen diese Zusammenhänge weniger deutlich oder sogar negativ ausgeprägt sind.


\pagebreak
\section{Ausblick}
Die vorliegenden Analysen liefern diverse Einblicke in das Pull-Request-Verhalten der Studierenden, insbesondere in den Projektmodul-Repositories. 
Auf Basis der gewonnenen Erkenntnisse ergeben sich verschiedene Punkte zur weiterführenden Forschung sowie der konkreten Erweiterung des \textit{GitGauge}-Tools. 

\subsection{Erweiterung von GitGauge}
Die im Rahmen dieser Arbeit untersuchten Metriken könnten teilweise direkt in GitGauge Auswertungen aufgenommen werden. Besonders interessant wären:
\begin{itemize}
\item \textbf{Automatisierte Klassifikation von Schliessgründen}: Die Kategorisierung der Pull Request Schliessgründe erfolgte in dieser Arbeit manuell. Eine automatisierte Erkennung auf Basis der Kommentaranalyse mittels eines \textit{LLMs} könnte implementiert werden. Dies würde Dozierenden eine zusätzliche Metrik zur Beurteilung der Teamkommunikation und des Review-Prozesses bieten.
\item \textbf{Integration von CI/CD-Metriken}: Derzeit werden CI/CD (\textit{Continuous Integration- und Deployment}) Metriken in GitGauge nicht berücksichtigt, da CI/CD in den Projektmodulen PM1, PM2 und PM3 nicht Teil des Lehrplans ist. Für eine erweiterte Nutzung des Tools, insbesondere für das Projektmodul 4 oder im professionellen Umfeld, wäre die Integration von Metriken wie Build-Zeit, Testdurchläufe oder Anzahl fehlgeschlagener Pipelines eine wertvolle Erweiterung.
\item \textbf{Analyse von Teamdynamiken}: Ergänzend zu den bestehenden Metriken könnten neue Visualisierungen zur Zusammenarbeit innerhalb der Teams implementiert werden. Beispielsweise könnten Netzwerkdiagramme auf Basis von Kommentaren aufzeigen, welche Teammitglieder miteinander interagieren. Ein Aktivitätsindex pro Person würde darüber hinaus helfen, individuelle Beiträge transparenter darzustellen. Solche Visualisierungen könnten zudem verdeutlichen, ob es innerhalb eines Teams isolierte Untergruppen (\textit{Inseln}) gibt oder ob eine gleichmässige Zusammenarbeit stattfindet.
\end{itemize}

\subsection{Weiterführende Forschung}
Aus den in dieser Arbeit durchgeführten Analysen ergeben sich weiterführende Fragestellungen, die über die aktuellen Projektmodule hinausgehen:
\begin{itemize}
\item \textbf{Nachhaltigkeit und weitere Metriken durch CI/CD}: Insbesondere für das Projektmodul PM4 oder externe Industrieprojekte könnte untersucht werden, wie sich \textit{CI/CD} auf Qualität und Reviewprozesse auswirkt. So könnte ein Fokus auf nachhaltige Praktiken gelegt werden. Beispielsweise indem Tests selektiv ausgeführt werden (z.,B. nur beim Öffnen eines Pull Requests anstatt bei jedem Commit), um Ressourcen zu schonen und die Effizienz zu steigern.
\item \textbf{Vergleich mit Enterprise-Projekten}: Eine Erweiterung des Vergleichs zwischen den Studierendenprojekten und professionellen Organisationen könnte weitere Unterschiede in Kultur, Formalisierung und Tooling aufdecken. Besonders spannend wären Analysen von grossen Organisationen, welche keine öffentliche Github Organisation besitzen.
\item \textbf{Einfluss von Teamstruktur und Berufserfahrung}: Die aktuell untersuchten Unterschiede zwischen Vollzeit- und Teilzeitstudierenden könnten noch detaillierter hinsichtlich Berufserfahrung, Vorwissen oder Teamkonstellation analysiert werden.
\end{itemize}


\subsection{Fazit}

Die Ergebnisse dieser Arbeit zeigen, dass Pull Requests mehr als nur ein technisches Tool im Entwicklungsprozess sind. Sie spiegeln auch die Dynamik von Teamprozessen, Zeitmanagement und Kommunikationsverhalten wider. Durch gezieltes Repository-Mining lassen sich daraus wertvolle Erkenntnisse für die Lehre, Projektsteuerung und Prozessverbesserung gewinnen.

Die analysierten Metriken liefern dabei konkrete Ansatzpunkte, um Qualität, Zusammenarbeit und Nachhaltigkeit zu messen.
Die vorliegenden Resultate bilden eine solide Basis für weiterführende Forschungsarbeiten, etwa zur Wirkung von CI/CD-Praktiken, zur Analyse von Teamstrukturen oder zum Vergleich mit industriellen Projekten.