% Indicate the main file. Must go at the beginning of the file.
% !TEX root = ../main.tex

%----------------------------------------------------------------------------------------
% EINLEITUNG
%----------------------------------------------------------------------------------------


\chapter{Theoretische Grundlagen} % Main chapter title

\label{Chapter2} % Change X to a consecutive number; for referencing this chapter elsewhere, use \ref{ChapterX}

%----------------------------------------------------------------------------------------
% SECTION 1
%----------------------------------------------------------------------------------------

\section{Git \& Github}
Die Nutzung eines Versionskontrollsystems (VCS) ist ein zentrales Element der Softwareentwicklung. Mit einem VCS können Änderungen am Sourcecode nachverfolgt und bei Bedarf eine frühere Version des Projektes wiederhergestellt werden. Die Nutzung eines VCS ermöglicht ausserdem, dass mehrere Entwickler parallel am gleichen Code arbeiten können, indem die Änderungen inklusive deren Urheber sowie Zeitpunkt festgehalten werden. 

Mit der Nutzung einer verteilten Versionsverwaltung (DVCS), wird kein zentrales Repository mehr verwendet. Jeder Entwickler verfügt über eine Kopie des Projektes inklusive des gesamten Projektverlaufs. Änderungen können lokal implementiert werden. Eine Verbindung zum Server wird nur für das Synchronisieren von Änderungen notwendig. \parencite{noauthor_informationen_2025} 

Git ist heute das am meisten verbreitete DVCS-System. Es wurde von Linus Torvalds für die Entwicklung des Linux-Kernels entwickelt. \parencite{zack_git_2018} Git bezeichnet ein Projekt als ein Git-Repository, wobei der Verlauf als eine Reihe von Snapshots (Commits) modelliert wird. Diese Commits sind in unterschiedlichen Branches organisiert, welche die parallele Entwicklung ermöglichen. Mit dem Verwenden von Branches kann parallel an Bugs, neuen Features etc. entwickelt werden, ohne sich gegenseitig zu stören. Diese unterschiedlichen Branches werden anschliessend gemerded. 

Mit dem Erfolg von Git entstanden Dienste und Anbieter, welche Git Repositories online hosten und zusätzliche Kollaborationstools anbieten. Github ist der bekannteste und auch am meist verwendete Hosting-Dienst. Auf Github können Entwickler ihr Git-Repository hosten, Änderungen vorschlagen und diskutieren, Fehler (Issues) tracken. Github bietet eine Weboberfläche, welche für das Einsehen des Codes, aber auch für das Erstellen von Pull Requests und Issues verwendet werden kann. Zusätzlich kann der CI/CD Prozess direkt mittels Github Actions abgebildet werden. \parencite{noauthor_informationen_2025}  

\section{Pull Request} 
Ein Pull Request ist ein Vorschlag, Änderungen von einem Branch oder Fork in einen anderen Branch zusammenzuführen. In einem typischen Entwicklungsworkflow erstellt der Entwickler für eine Änderung einen neuen Branch, committet dort seine Arbeit und erstellt anschliessend einen Pull Request. Ein Pull Request enthält dabei die durchgeführten Codeänderungen sowie eine Beschreibung zur Änderung. Die Unterschiede (Diffs) zwischen Quell- und Zielbranch werden dabei übersichtlich aufgezeigt \parencite{noauthor_about_nodate}.

Pull Requests spielen für den Code-Review und die Diskussion innerhalb des Entwicklungsprozesses eine wichtige Rolle. So bietet ein Pull Request eine Basis zur Diskussion, indem Teammitglieder den Code begutachten, kommentieren und auch Änderungen vorschlagen können \parencite{atlassian_pull_nodate}. Viele Entwicklerteams richten daher verbindliche Review-Prozesse ein, bei denen ein Pull Request (PR) erst dann gemergt werden kann, wenn dieser von mindestens einer Person genehmigt wurde \parencite{jiang_how_2022}.

Durch das Konfigurieren von automatischen Checks mittels CI/CD können sowohl Anforderungen an die Softwarequalität (zum Beispiel durch das automatisierte Ausführen von Tests oder statischen Codeanalysen) als auch an den Entwicklungsprozess automatisiert überprüft werden \parencite{kinsman_how_2021}. Dazu gehört beispielsweise die Überprüfung, ob alle Commits mit einem Signed-off-by versehen sind, was in vielen Projekten zur Einhaltung des Developer Certificate of Origin (DCO) verpflichtend ist \parencite{holtgrave_attributing_2025}. Diese Massnahmen stellen sicher, dass sowohl die technische Qualität des Codes als auch rechtliche Rahmenbedingungen eingehalten werden, bevor ein Pull Request gemergt wird \parencite{noauthor_about_nodate}. 

\section{Pull Request Akzeptanz}
Am Ende eines Pull Requests wird dieser entweder akzeptiert und in den Haupt-Branch integriert (Merged) oder geschlossen werden (Close)\parencite{noauthor_merging_nodate}\parencite{noauthor_closing_nodate}. Die Akzeptanz kann dabei auf zwei Arten erfolgen. Entweder wird der Merge-Button durch den Antragsteller oder ein anderes Teammitglied direkt angeklickt oder es erfolgt eine Genehmigung (Approval) \parencite{noauthor_merging_nodate}\parencite{noauthor_reviewing_nodate}. Die Genehmigung ist eine Aktion welche auf einem Pull Request ausgeführt werden kann. Es besteht auch die Möglichkeit, diese Aktion zu erzwingen und somit dürfen die Änderungen erst zusammengeführt werden, wenn ein anderes Teammitglied den Pull Request genehmigt hat \parencite{noauthor_approving_nodate}.

\section{Branching Strategien}
Es existieren verschiedene Branching-Strategien. Die vier bekanntesten sind der Git Flow, GitHub Flow, GitLab Flow und Trunked-based Development. \parencite{priyanka_gowdaashwath_narayana_gowda_git-branching-and-release-strategies_2022} \parencite{atlassian_trunk-based_nodate}
\subsection{Git Flow}
Diese Strategie besteht aus zwei Hauptbranches:
\begin{itemize}
    \item Master: Beinhaltet den produktionsreifen Code  
    \item Develop: Integrationsebene für neue Funktionen
\end{itemize}
und drei Hilfsbranches:
\begin{itemize}
    \item Feature: Entwicklung von neuen Funktionen
    \item Hotfix: Für kritische Fehler in der Produktion
    \item Release: Vorbereitungsebene für Funktionen die produktiv werden sollen.
\end{itemize}
Neue Funktionen werden in Feature Branches entwickelt und nach Beendigung in den Develop Branch gemerged. Der Develop Branch wird dann in den jeweiligen Release Branch und schlussendlich in den main Branch gemerged. Falls es in der Produktion zu einem kritischen Fehler kommt, welcher nicht bis zum neuen Release warten kann wird jener in einem Hotfix Branch behoben und zurück in den Main gemerged.
\parencite{priyanka_gowdaashwath_narayana_gowda_git-branching-and-release-strategies_2022}

\subsection{GitHub Flow}
Die GitHub Flow Strategie ist eine vereinfachte und leichtgewichtige Methode. Sie besteht aus den folgenden Branches:
\begin{itemize}
    \item Main: Beinhaltet den produktionsreifen Code  
    \item Feature: Entwicklung von neuen Funktionen
\end{itemize}
Auch in dieser Strategie werden neue Funktionen in einem Feature Branch entwickelt. Diese werden jedoch nach Abschluss, via Pull Requests, direkt in den Main gemerged.
\parencite{priyanka_gowdaashwath_narayana_gowda_git-branching-and-release-strategies_2022}

\subsection{GitHub Flow}
Der GitLab Flow befindet sich zwischen der Git- und der GitHub-Flow-Methode. 
\begin{itemize}
    \item Production: Beinhaltet den produktiven Code  
    \item Staging: Umgebungsspezifischeebene
    \item Development: Integrationsebene für neue Funktionen
    \item Feature: Entwicklung von neuen Funktionen
\end{itemize}
Neue Funktionen werden ebenfalls in den Feature Branches entwickelt. Diese werden dann in den Development Branch gemerged. Danach werden sie in die entsprechenden Staging Branches gemerged und sobald sie produktionsreif sind, in den Production Branch gemerged.
\parencite{priyanka_gowdaashwath_narayana_gowda_git-branching-and-release-strategies_2022}

\subsection{Trunked-based development}
Diese Strategie basiert auf kurzlebigen Branches, welche dann in den Main-Branch integriert werden.
\begin{itemize}
    \item Main: Beinhaltet den Produktionsreifen Code 
    \item Kurzlebiger Branch: Kleine Änderungen, keine ganzen Features
\end{itemize}
Bei dieser Strategie werden kleine Änderungen über kurzlebige Branches in den Main Branch gebracht. Dabei werden nicht ganze Features in einem Branch implementiert, sondern diese werden in kleine Stückchen aufgeteilt.
\parencite{atlassian_trunk-based_nodate}

\section{Projektmodule}
\label{sec:Projektmodule} 
An der ZHAW finden im Verlauf des Informatikstudiums vier Projektmodule (PM) statt. In diesen Modulen geht es darum, Erfahrungen im Management und der Realisierung von Softwareprojekten im Team zu sammeln \parencite{noauthor_modul_nodate}. 

Im Rahmen des Projektmodul 1 werden in Vierergruppen drei Projekte in jeweils drei Wochen durchgeführt. In diesen Projekten wird bereits mit Git gearbeitet, das Arbeiten mit Pull Requests wird jedoch noch nicht forciert und ist daher für diese Studie irrelevant. Ab dem PM2 werden Pull Requests vorausgesetzt. Das Projektmodul hat sich im Laufe der analysierten Repositories verändert. In der früheren Version werden zwei Projekte in Vierergruppen in jeweils vier Wochen und ein Projekt in einer Zweiergruppe innerhalb von drei Wochen durchgeführt. Dabei wird im ersten Projekt das Pen \& Paper Spiel Racetrack als textbasiertes Spiel umgesetzt. Im zweiten Projekt erhalten die Studierenden ein fertiges Chatprogramm, welches analysiert, verbessert und repariert werden muss. Als letztes Projekt soll eine kleine JavaFX-Applikation zu einem selbst gewählten Thema umgesetzt werden. In der späteren Version wurde das zweite Projekt, das Chatprogramm, welches 3 Wochen dauert, entfernt und dafür das eigene Projekt um 2 Wochen verlängert. Im dritten Projektmodul wird innerhalb eines Semesters in zufällig ausgewählten Gruppen von circa fünf Personen ein selbst gewähltes Projekt umgesetzt. Dabei geht es darum, ein Projekt von der Vision bis zur lauffähigen Anwendung in einem iterativen und inkrementellen Prozess der Software-Entwicklung zu realisieren. Das letzte Projektmodul umfasst ebenfalls die Umsetzung eines eigenständig gewählten Projekts innerhalb eines Semesters. Im Unterschied zu PM3 stellen die Studierenden eigene Teams in der Grösse von 7-8 Personen zusammen und es werden CI/CD-Praktiken erforderlich.

Der Kenntnisstand der Studierenden ist unterschiedlich. Die ZHAW verlangt mindestens ein Jahr Berufserfahrung in einem verwandten Beruf vor Studienbeginn oder ein spezielles Praktikumsprogramm während des Studiums \parencite{noauthor_aufnahmebedingungen_nodate}. So gibt es Studierende mit mehrjähriger Erfahrung in der Durchführung von Softwareprojekten und solche ohne jegliche Erfahrung. 

Der Studiengang Informatik wird sowohl als Vollzeit- als auch als Teilzeitstudium angeboten. Die Dauer eines Vollzeitstudiums beläuft sich auf sechs Semester, während das Teilzeitstudium eine Dauer von acht Semestern beträgt. Bei den Teilzeitstudierenden kann aufgrund von persönlichen Erfahrungen der Autoren davon ausgegangen werden, dass die Mehrheit im Bereich der Informatik arbeitet. Das Studium wird an den Standorten Winterthur und Zürich durchgeführt. Dabei unterscheiden sich die Unterrichtsmodelle bei den Teilzeitstudierenden an den zwei Standorten. In Winterthur wird ein Tagesmodell angeboten. Dies bedeutet, dass der Unterricht an zwei vollen Arbeitstagen und am Samstagvormittag durchgeführt wird. In Zürich gilt ein Mischmodell, dort findet der Unterricht an einem Arbeitstag und an zwei Abenden statt. Dies hat zur Folge, dass die Projektmodule für die verschiedenen Klassen an unterschiedlichen Tagen erfolgen. Sie finden jedoch für alle Studierenden im selben Semester statt. \parencite{noauthor_bachelorstudium_nodate}\parencite{noauthor_teilzeitstudium_nodate}

\section{ Open Source Projekte}
Vergleicht man die Git-Repositories der Projektmodule mit Open-Source-Projekten, so lassen sich wesentliche Unterschiede feststellen. Während die Lebensdauer eines PM-Git-Repositories strikt auf ein Semester begrenzt ist, sind Open-Source-Projekte in der Regel langfristig angelegt und haben keinen festen Endpunkt. Viele dieser Projekte entwickeln sich über Jahre oder gar Jahrzehnte hinweg weiter, da ihre Entwicklungsziele flexibel angepasst werden.

Ein weiterer Unterschied liegt im Entwicklungstempo. Open-Source-Projekte folgen keinem festen Zeitplan, sondern entwickeln sich in einem unregelmässigen Rhythmus, da die Beteiligung stark von der Verfügbarkeit der Maintainers abhängt. Gewisse Features werden schnell implementiert, während andere über längere Zeiträume hinweg stagnieren. In akademischen Projektmodulen hingegen ist der Arbeitsaufwand gleichmässiger verteilt, da die Studierenden festen Abgabefristen folgen müssen. Dies führt zu einer planbaren, wenn auch vergleichsweise langsamen Entwicklungsgeschwindigkeit.

In den OpenSource-Projekten spielen die User-Metriken des Contributors eine wichtige Rolle. So wird beispielsweise die Anzahl der Followers des Entwicklers als ein Hinweis auf seinen Status und Ruf in der Community gewertet. Untersuchungen zeigen, dass Beiträge von Nutzern mit vielen Followern eine höhere Chance, akzeptiert zu werden (pro Standardabweichung an zusätzlichen Followern steigt die Wahrscheinlichkeit der Merge-Annahme um 18\%). Auch projektbezogene Metriken wie eine hohe Anzahl Stars des Projektes spielen hierbei eine Rolle. \parencite{tsay_influence_2014}
