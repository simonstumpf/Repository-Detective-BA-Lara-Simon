% Indicate the main file. Must go at the beginning of the file.
% !TEX root = ../main.tex

%----------------------------------------------------------------------------------------
% EINLEITUNG
%----------------------------------------------------------------------------------------


\chapter{Theoretische Grundlagen} % Main chapter title

\label{Chapter2} % Change X to a consecutive number; for referencing this chapter elsewhere, use \ref{ChapterX}

%----------------------------------------------------------------------------------------
% SECTION 1
%----------------------------------------------------------------------------------------

\section{Git \& Github}
Die Nutzung eines Versionskontrollsystems (VCS) ist ein zentrales Element der Softwareentwicklung. Mit einem VCS können Änderungen am Sourcecode nachverfolgt und bei Bedarf eine frühere Version des Projektes wiederhergestellt werden. Die Nutzung eines VCS ermöglicht ausserdem, dass mehrere Entwickler parallel am gleichen Code arbeiten können, indem die Änderungen inklusive deren Urheber sowie Zeitpunkt festgehalten werden. 

Mit der Nutzung einer verteilten Versionsverwaltung (DVCS), wird kein zentrales Repository mehr verwendet. Jeder Entwickler verfügt über eine Kopie des Projektes inklusive des gesamten Projektverlaufs. Änderungen können lokal implementiert werden. Eine Verbindung zum Server wird nur für das Synchronisieren von Änderungen notwendig. \parencite{noauthor_informationen_2025} 

Git ist heute das am meist verbreitete DVCS System. Es wurde von Linus Torvalds für die Entwicklung des Linux-Kernels entwickelt. \parencite{zack_git_2018} Git bezeichnet ein Projekt als ein Git-Repository, wobei der Verlauf als eine Reihe von Snapshots (Commits) modelliert wird. Diese Commits sind in unterschiedlichen Branches organisiert, welche die parallele Entwicklung ermöglichen. Mit dem Verwenden von Branches kann parallel an Bugs, neuen Features etc. entwickelt werden, ohne sich gegenseitig zu stören. Diese unterschiedlichen Branches werden anschliessend gemerdet. 

Mit dem Erfolg von Git entstanden Dienste und Anbieter, welche Git Repositories online hosten und zusätzliche Kollaborationstools anbieten. Github ist der bekannteste und auch am meist verwendete Hosting-Dienst. Auf Github können Entwickler ihr Git-Repository hosten, Änderungen vorschlagen und diskutieren, Fehler (Issues) tracken. Github bietet eine Weboberfläche, welche für das Einsehen des Codes, aber auch für das Erstellen von Pull Requests und Issues verwendet werden kann. Zusätzlich kann der CI/CD Prozess direkt mittels Github Actions abgebildet werden. \parencite{noauthor_informationen_2025}  

\section{Pull Request} 
Ein Pull Request ist ein Vorschlag, Änderungen von einem Branch oder Fork in einen anderen Branch zusammenzuführen. In einem typischen Entwicklungsworkflow erstellt der Entwickler für eine Änderung einen neuen Branch, committet dort seine Arbeit und erstellt anschliessend einen Pull Request. Ein Pull Request enthält dabei die durchgeführten Codeänderungen sowie eine Beschreibung zur Änderung. Die Unterschiede (Diffs) zwischen Quell- und Zielbranch werden dabei übersichtlich aufgezeigt \parencite{noauthor_about_nodate}.

Pull Requests spielen für den Code-Review und die Diskussion innerhalb des Entwicklungsprozesses eine wichtige Rolle. So bietet ein Pull Request eine Basis zur Diskussion, indem Teammitglieder den Code begutachten, kommentieren und auch Änderungen vorschlagen können \parencite{atlassian_pull_nodate}. Viele Entwicklerteams richten daher verbindliche Review-Prozesse ein, bei denen ein Pull Request (PR) erst dann gemergt werden kann, wenn dieser von mindestens einer Person genehmigt wurde \parencite{jiang_how_2022}.

Durch das Konfigurieren von automatischen Checks mittels CI/CD können sowohl Anforderungen an die Softwarequalität (zum Beispiel durch das automatisierte Ausführen von Tests oder statischen Codeanalysen) als auch an den Entwicklungsprozess automatisiert überprüft werden \parencite{kinsman_how_2021}. Dazu gehört beispielsweise die Überprüfung, ob alle Commits mit einem Signed-off-by versehen sind, was in vielen Projekten zur Einhaltung des Developer Certificate of Origin (DCO) verpflichtend ist \parencite{holtgrave_attributing_2025}. Diese Massnahmen stellen sicher, dass sowohl die technische Qualität des Codes als auch rechtliche Rahmenbedingungen eingehalten werden, bevor ein Pull Request gemergt wird \parencite{noauthor_about_nodate}. 

\section{Pull Request Akzeptanz}
Am Ende eines Pull Requests wird dieser entweder akzeptiert und in den Haupt-Branch integriert (Merged) oder geschlossen werden (Close)\parencite{noauthor_merging_nodate}\parencite{noauthor_closing_nodate}. Die Akzeptanz kann dabei auf zwei Arten erfolgen. Entweder wird der Merge-Button durch den Antragsteller oder ein anderes Teammitglied direkt angeklickt oder es erfolgt eine Genehmigung (Approval) \parencite{noauthor_merging_nodate}\parencite{noauthor_reviewing_nodate}. Die Genehmigung ist eine Aktion welche auf einem Pull Request ausgeführt werden kann. Es besteht auch die Möglichkeit, diese Aktion zu erzwingen und somit dürfen die Änderungen erst zusammengeführt werden, wenn ein anderes Teammitglied den Pull Request genehmigt hat \parencite{noauthor_approving_nodate}.

