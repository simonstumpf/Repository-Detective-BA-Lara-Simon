% Indicate the main file. Must go at the beginning of the file.
% !TEX root = ../main.tex

%----------------------------------------------------------------------------------------
% CHAPTER TEMPLATE
%----------------------------------------------------------------------------------------


\chapter{Stand der Forschung} % Main chapter title

\label{Chapter3} % Change X to a consecutive number; for referencing this chapter elsewhere, use \ref{ChapterX}

%----------------------------------------------------------------------------------------
% SECTION 1
%----------------------------------------------------------------------------------------

\section{Pull Request Dauer (Lara)}
Es existieren bereits diverse Studien, die sich mit der Analyse von Pull Requests befassen. In der Literatur werden verschiedene Faktoren untersucht, die beispielsweise die Dauer von Pull Requests bestimmen. Ein Beispiel ist das Paper "Wait for It: Determinants of Pull Request Evaluation Latency on GitHub", welches relevante Faktoren bei GitHub-Projekten analysiert hat. Die Ergebnisse der Untersuchung legen nahe, dass die Faktoren, die den Pull Request selbst betreffen, wie die Grösse (d. h. die Anzahl der Änderungen des Pull Requests) oder die Länge der Diskussion über den Pull Request, sowie Entwicklerfaktoren wie seine Reputation, einen Einfluss auf die Dauer des Pull Request haben. Zusätzlich konnte ein Zusammenhang zwischen der initialen Reaktion und den Wartezeiten festgestellt werden. Ausserdem wurde das Warten auf automatisierte Tests als ein weiterer wichtiger Faktor identifiziert.\\
Ein weiteres Paper befasst sich spezifischer mit dem Punkt der ersten Reaktion auf einen Pull Request anhand Open Source Projekten. Es wurde festgestellt, dass die erste Antwort eines Bots kaum einen Einfluss auf die Dauer hat, die erste Antwort eines Menschen jedoch einen grossen Einfluss hat. Die Dauer der ersten menschlichen Reaktion ist auf die oben genannten Faktoren zurückzuführen.

%----------------------------------------------------------------------------------------
% SECTION 2
%----------------------------------------------------------------------------------------

\section{Pull Request Akzeptanz}
Simon
