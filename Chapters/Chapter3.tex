% Indicate the main file. Must go at the beginning of the file.
% !TEX root = ../main.tex

%----------------------------------------------------------------------------------------
% CHAPTER TEMPLATE
%----------------------------------------------------------------------------------------


\chapter{Methodik} % Main chapter title

\label{Chapter3} % Change X to a consecutive number; for referencing this chapter elsewhere, use \ref{ChapterX}

%----------------------------------------------------------------------------------------
% SECTION 1
%----------------------------------------------------------------------------------------
Dieses Kapitel beschreibt die Methoden und Vorgehensweisen, zur Beantwortung der Forschungsfragen. Ziel ist es, aussagekräftige Metriken zu extrahieren und Zusammenhänge festzustellen. 
In diesem Kapitel werden die Metriken vorgestellt, die zur Beantwortung der Forschungsfragen notwendig sind. Es werden Definitionen und Berechnungen vorgestellt, um ein Verständnis für diese Metriken zu erlangen. Des Weiteren wird aufgezeigt, welche Metriken mittels Mining-Verfahren aus den Repositories extrahiert werden. Abschliessend wird die Datengrundlage vorgestellt, auf der die Analysen basieren.

\section{Erarbeitung der Forschungsfragen}
\label{sec:ErarbeitungFF}

\section{Vorgehensweise bei den Forschungsfragen}
\label{sec:AnalyseChurnvsLatency}
\label{sec:Metriken}
Um die Forschungsfragen beantworten zu können, musste das Mining erweitert werden. Dafür wurden weitere GraphQL-Abfragen implementiert und ein neuer  Miner für Repositories erstellt sowie der Miner für Pull Requests erweitert. Bei den Minern handelt es sich um sogenannte Extractor-Services, welche die GraphQL-Abfragen implementieren und die Daten in den entsprechenden Entitäten abspeichern.

Um die erste \fref{forschungsfrage1} beantworten zu können, müssen die Metriken \textit{Latency} und \textit{Churn} eingeführt werden. Die \textit{Latency} bestimmt die Zeitspanne zwischen der Eröffnung und der Schliessung eines Pull Requests, sei es durch das Mergen des PRs oder durch manuelle Schliessung. Die manuelle Schliessung bedeutet, dass der Pull Request entweder abgelehnt wurde oder durch den Autor des PRs erfolgte. Die \textit{Latency} wird anhand folgender Formel berechnet:
\begin{equation}
latency = closeTime - createTime
\end{equation}
Dabei handelt es sich bei den Metriken \textit{closeTime} und \textit{createTime} um zwei Metriken, die mittels Mining gewonnen werden können. Diese sind bereits im Issue-Miner von GitGauge umgesetzt und mussten nicht neu implementiert werden. Bei den Namen der Metriken handelt es sich um die Namen der Attribute, wie sie in Git Gauge genannt werden.

Des Weiteren wird die Metrik \textit{Churn} gebraucht. Diese Metrik setzt sich aus der Anzahl geänderter Zeilen im Pull Request zusammen \parencite{gousios_exploratory_2014}. Folgend die Formel für \textit{Churn} (die Namen der Metriken werden von den Namen der Attribute in GitGauge abgeleitet): 
\begin{equation}
churn = additions + deletions
\end{equation}

Für die Metriken \textit{additions} und \textit{deletions} musste das Mining erweitert werden. Dazu wurde die bestehende Pull Request-Entität um die Metriken \textit{additions} und \textit{deletions} ergänzt und im Miner für Pull Request eine Abfrage via GraphQL erstellt. Die entsprechende GraphQL-Abfrage ist in der folgenden \autoref{fig:GraphQL-Abfrage-additions-deletions} ersichtlich. Dabei konnten die Metriken direkt von der Pull Request-Entität der entsprechenden GraphQL-Bibliothek (Octokit.GraphQL) bezogen werden.

\begin{lstlisting}[language=CSharp, caption={GraphQL-Abfrage additions und deletions}]
var query = new Query()
.Repository(repositoryData.Repository!.Project, repositoryData.Repository!.Owner)
.PullRequests()
.AllPages()
    .Select(pr => new PullRequest
    {
        GitHubId = pr.Id.Value,
        Number = pr.Number,
        Title = pr.Title,
        Description = pr.BodyText,
        CreateTime = pr.CreatedAt,
        UpdateTime = pr.UpdatedAt,
        CloseTime = pr.ClosedAt,
    ......
}).Compile();
\end{lstlisting}

Um Anschliessend den Zusammenhang der Metriken analysieren zu können, wird die Rangkorrelation nach Spearman verwendet. Mittels der Rangkorrelation nach Spearman \parencite{noauthor_t-test_nodate} kann überprüft werden, ob ein Zusammenhang zwischen zwei Variablen besteht. Dabei verwendet Spearman die Rangplätze der Daten und nicht deren Wert. Der daraus resultierende Korrelationskoeffizient r\textsubscript{s} kann zwischen -1 und 1 liegen, wobei dieser bei eins ein starker positiver Zusammenhang und bei -1 ein starker negativer Zusammenhang zwischen den zwei Variablen besteht. Wenn der Korrelationskoeffizient bei null liegt, besteht keine Korrelation. \parencite{noauthor_t-test_nodate}  \\
Die Spearman-Formel \parencite{noauthor_t-test_nodate} stellt sich wie folgt zusammen: 
\begin{equation}
r_s = 1 - \frac{6 \sum d_i^2}{n(n^2 - 1)}
\end{equation}
\label{eqn:spearman}
\noindent\textbf{Legende:}
\begin{itemize}
  \item [$r_s$] Spearman-Rangkorrelationskoeffizient
  \item[$d_i$] Differenz der Ränge zwischen den zwei Variablen 
  \item[$n$] Anzahl der beobachteten Fälle
\end{itemize}

Zur Beantwortung der \fref{forschungsfrage2} ist der Status des Pull Requests erforderlich. Es sei darauf hingewiesen, dass diese Metrik nicht unmittelbar von GitHub extrahiert werden kann, sondern dass diese ermittelt werden muss. Dazu werden die Metriken \textit{closeTime} und \textit{mergeTime} benötigt. Im ersten Schritt wird überprüft, ob der PR  eine \textit{closeTime} aufweist. Ist dies der Fall, wird im zweiten Schritt überprüft, ob ebenfalls eine \textit{mergeTime} vorhanden ist. Somit kann evaluiert werden, ob der Pull Request geschlossen wurde und ob dies durch ein Merging erfolgte. Diese Überprüfung ist bereits in GitGauge implementiert und wird entsprechend zur Beantwortung der \fref{forschungsfrage2} verwendet. Im Falle der \fref{forschungsfrage2} interessieren die Pull Requests, die geschlossen, aber nicht gemergt wurden. Die Analyse der entsprechend geschlossenen PRs wurde manuell durchgeführt.

Bei \fref{forschungsfrage3} wird zusätzlich zu den bereits erwähnten Metriken \textit{createTime} und \textit{closeTime} des Pull Requests die \textit{createTime} des Repositories benötigt, um berechnen zu können, zu welchem Zeitpunkt im Projekt ein Pull Request erstellt oder geschlossen wurde.  Dafür wurde im neuen Repository-Miner die \textit{createTime } mittels GraphQL-Abfrage extrahiert. Dies ist in der \autoref{fig:GraphQl-Abfrage-repo} ersichtlich. Zudem wurde die Repository-Entität um das Erstellungsdatum erweitert. 
\begin{lstlisting}[language=CSharp, caption={GraphQL-Abfrage Repository}]
var repositoryQuery = new Query()
    .Repository(owner: repositoryData.Repository.Owner, name: repositoryData.Repository.Project)
    .Select(repo => new
    {
        repo.Name,
        repo.Owner.Login,
        repo.Description,
        repo.CreatedAt
    }).Compile();
\end{lstlisting}

Zusätzlich wurden die Abgabetermine genutzt, um zu analysieren, wie weit ein Ereignis von der Abgabe entfernt erfolgte. Diese Termine wurden für die Projekte bereitgestellt und konnten nachgeschlagen werden.

Zur Identifizierung von Mustern in den Pull Requests, die in der \fref{forschungsfrage4} thematisiert werden, erfolgt zusätzlich zu den Metriken \textit{createTime} und \textit{closeTime} eine Analyse der Commits auf den Pull Requests. Zu diesem Zweck wurde die Pull Request-Entität um die Commit-Entität erweitert. Die Commits beinhalten die Attribute \textit{Sha}, \textit{Message}, \textit{CommitTime} und \textit{Author}. Beim \textit{Sha} handelt es sich um eine Id, mit welcher der Commit eindeutig identifiziert werden kann. Die \textit{Message} beinhaltet die Commit-Message, die bei einem Commit angegeben werden muss. Mittels der \textit{CommitTime} ist ersichtlich, wann der Commit gemacht wurde, und der \textit{Author} zeigt, wer den Commit gemacht hat. Zusätzlich musste die GraphQL-Abfrage im Pull Request-Miner erweitert werden. Dies ist in \autoref{fig:GraphQl-Abfrage-commits} ersichtlich. Dafür wurden aus der Pull Request-Entität der GraphQL-Bibliothek die Commits ausgelesen und die entsprechenden Parameter abgefüllt.
\begin{lstlisting}[language=CSharp, caption={GraphQL-Abfrage Pull Request Commits}, label={lst:graphql-commits}]
PullRequestCommits = pr.Commits(null, null, null, null)
    .AllPages()
    .Select(commit => new PullRequestCommitEntity()
    {
        PullRequestGraphqlId = commit.PullRequest.Id.Value,
        GitHubId = commit.Commit.Oid,
        Sha = commit.Commit.Oid,
        Message = commit.Commit.Message,
        CommitTime = commit.Commit.CommittedDate,
    }).ToList(),
\end{lstlisting}

\newpage
Die bereits aufgeführten Erweiterungen genügten für die \fref{forschungsfrage5}, sodass keine zusätzlichen Änderungen im Mining von GitGauge vorgenommen werden mussten. Jedoch mussten bei der Analyse die Klassen auf Vollzeit und Teilzeit aufgeteilt werden. In den Projektmodulen wird für jede Klasse jeweils eine GitHub-Organisation erstellt. Anhand der Namen dieser Organisationen kann erkannt werden, ob es sich um eine Teilzeit- oder Vollzeitklasse handelt, da der Klassenname darin vorkommt. Die Klassennamen für Informatik setzen sich wie folgt zusammen: 
\begin{equation}
IT<yy><Klassenkuerzel><Ort>
\end{equation}
\noindent\textbf{Legende:}
\begin{itemize}
  \item \textit{yy}:  DIe letzten zwei Ziffern des Jahres bei Studiebegninn (z.B. 23)
  \item\textit{Klassenkuerzel}: Bei Vollzeitklassen: a/b bei Teilzeit ta/tb
  \item\textit{Ort}: WIN/ZH (Abkürzungen für Winterthur und Zürich)
\end{itemize}

Anhand des \textit{t}'s kann somit erkannt werden, ob eine Klasse Teilzeit oder Vollzeit ist.

Für die \fref{forschungsfrage6} wurden sieben Metriken festgelegt, anhand derer die Korrelationsanalyse und der Vergleich mit OSS-Projekten stattfinden sollen. Die sieben Metriken sind: \textit{Latency}, \textit{Churn}, \textit{Commits}, \textit{Comments}, \textit{Description Length}, \textit{Changed Files Count} und \textit{Contributors}. Diese wurden, wie in Kapitel \secref{sec:ErarbeitungFF} erwähnt, durch Literaturrecherche festgelegt. Die entsprechenden Metriken werden im folgenden Abschnitt \secref{sec:MetrikenKorrelation} aufgezeigt und die entsprechenden OSS-Projekte werden im Kapitel \secref{sec:VorstellungGithubOrgs} vorgestellt. Für die Korrelationsanalyse musste das Mining erweitert werden. Zusätzlich zu den bereits genannten Erweiterungen wurden noch die Metriken \textit{ChangedFiles} und \textit{TotalCommits} extrahiert.

Die Analysen der Forschungsfragen erfolgten via Jupyter Notebooks. Für die einzelnen Untersuchungen wurde jeweils ein neues Jupyter Notebook erstellt. Diese ermöglichen eine schnelle und unkomplizierte (graphische) Auswertung der Daten. \parencite{noauthor_repo-detectivesba-metric-analysis-scripts_nodate}


\section{Auswahl Metriken für Korrelationsanalyse}
\label{sec:MetrikenKorrelation}
Für die Korrelationsanalyse der Metriken zur Klärung von \fref{forschungsfrage6} müssen geeignete Metriken ausgewählt werden, die verschiedene Aspekte der Pull-Request-Dynamik abbilden. Soziale Faktoren wie die Erfahrung der Contributor oder die Dauer ihrer Mitarbeit an einem Projekt können jedoch nicht berücksichtigt werden, da solche Daten im Kontext der untersuchten Studenten-Repositories nicht verfügbar sind. Basierend auf der Analyse diverser wissenschaftlicher Literatur wurden folgende sieben Metriken ausgewählt:

Als \textbf{Latency} wird die Zeit zwischen der Erstellung eines Pull Requests und dessen Merge oder Schliessung genannt. Sie wird als Indikator für die Effizienz des Review-Prozesses genutzt und ist in der Literatur häufig als Schlüsselfaktor für die Messung der Geschwindigkeit von Code-Reviews und DevOps-Prozessen anerkannt. Eine hohe Latenz kann auf ineffiziente Reviews oder Verzögerungen hinweisen, weshalb sie in die Analyse aufgenommen wurde, um die Geschwindigkeit der Pull-Request-Bearbeitung zu messen und Zusammenhänge mit anderen Metriken wie \textit{Churn} und \textit{Comments} zu untersuchen. \parencite{yu_wait_2015}

Die Metrik \textbf{Churn} misst die Anzahl geänderter Codezeilen in einem Pull Request. Sie wird als Indikator für den Umfang und die Komplexität einer Änderung betrachtet und ist häufig mit längeren Review-Zeiten verbunden. Diese Metrik wurde ausgewählt, da grössere Änderungen im Code oft genauere Überprüfungen erfordern und somit die \textit{Latency} des Pull Requests beeinflussen können. \parencite{gousios_exploratory_2014}

Die Anzahl der \textbf{Commits} in einem Pull Request gibt an, wie viele Commits gemerdet werden sollen. Sie zeigt wie kleinschrittig oder iterativ ein Beitrag entwickelt wurde. Mehrere Commits deuten häufig darauf hin, dass der Pull Request mehrere Anpassungen durchlaufen hat, was zu einer längeren Review-Dauer führt. \parencite{zhang_pull_2022}

\textbf{Comments} messen die Anzahl der Kommentare, die während des Review-Prozesses als auch auf den Pull Request selber erstellt wurden. Eine höhere Anzahl an Kommentaren deutet auf intensivere Diskussionen hin, die den Review-Prozess verlängern können und die Wahrscheinlichkeit eines Merges verringern. Zudem ist eine hohe Zahl an Kommentaren oft mit einer höheren Qualität und Gründlichkeit des Reviews verbunden. \parencite{tsay_influence_2014}

Die Länge der Beschreibung eines Pull Requests (\textbf{Description Length}) ist ein Indikator für den Kontext, den der Entwickler für seine Änderungen liefert. Ausführliche Beschreibungen helfen den Reviewern, die Änderungen schneller zu verstehen, was den Review-Prozess beschleunigen kann. Kürzere Beschreibungen führen häufig zu mehr Rückfragen und somit zu einer längeren Review-Dauer. \parencite{zhang_pull_2022}

Neben der Anzahl geänderter Codezeilen gibt \textbf{Changed Files Count} an, wie viele Dateien von einer Änderung betroffen sind. Diese Metrik ist wichtig, da Änderungen, die viele Dateien betreffen, in der Regel komplexer sind und daher länger dauern, um überprüft und integriert zu werden. \parencite{tsay_influence_2014}

Die Metrik \textbf{Contributors} misst die Anzahl der verschiedenen Entwickler, die an einem Pull Request beteiligt sind. Ein höherer Beitrag von verschiedenen Entwicklern könnte auf eine intensivere Zusammenarbeit hinweisen und könnte mit der Komplexität und der Dauer des Review-Prozesses korrelieren.

    
\section{Architektur}

\section{Erkennung Squashing}
Mit \textit{Git Squashing} können mehrere Commits zu einem Commit zusammengeführt werden, bevor diese in den Zielbranch gemerged werden. Die theoretischen Grundlagen werden in \autoref{subsec:GitSquashing} erläutert. Die Implementierung des GitGauge-Features der Metrik der Commits pro Contributer (visualisiert als Timeline pro Wochentag sowie als Projektverlauf), bietet die Option gesqushte Feature Commits zu inkludieren (\textit{Include Squashed Feature Commits}). Ziel dieses Feature ist es, den Arbeitsverlauf von gesquashten Feature-Branches in die Grafik mit einzufliessen, da diese ebenfalls wichtig für die Gesamtbetrachtung sind. 
Je nach Einstellung des Feature-Toggles wird die Datenverarbeitung wie folgt angepasst: \\
\begin{itemize}
    \item \textbf{Feature deaktiviert}: Es werden ausschliesslich die Commits aus dem Main-Branch berücksichtigt. Gesquashte Commits aus Feature-Branches werden als ein Commit analyisiert.
    \item \textbf{Feature aktiviert}: Zwei Filter- und Einbeziehungsschritte werden vorgenommen:
        \begin{itemize}
        \item Gesquashte Commits auf dem Main-Branch werden ausgeschlossen.
        \item Alle Commits aus den gesquashten Feature-Branches werden in die Analyse integriert.
    \end{itemize}
\end{itemize}

Die nicht gesquashten Pull Requests müssen dabei nicht inkludiert werden, da die gesamte Feature-Branch Historie auf dem Main-Branch vorhanden ist. 

Die folgende Logik zeigt die Verarbeitung gesquashter Commits in der Methode \textit{GetCommitsCountPerDate} der Klasse \textit{ContributorMetricsCalculator}. Diese Vorgehensweise wurde konsistent in allen Methoden hinzugefügt, welche zur Berechnung der Graphen erforderlich sind.

\begin{lstlisting}[language=CSharp, caption={Verarbeitung gesquashter Commits in der Methode \textit{GetCommitsCountPerDate}}, label={lst:graphql-commits}]
void processCommit(DateTime commitDate, ContributorDto contributor)
{

	if (!commitsCountPerDate.ContainsKey(commitDate))
	{
		commitsCountPerDate[commitDate] = new Dictionary<String, int>();
	}

	if (!commitsCountPerDate[commitDate].ContainsKey(contributor.Username))
	{
		commitsCountPerDate[commitDate][contributor.Username] = 0;
	}

	commitsCountPerDate[commitDate][contributor.Username]++;
}
foreach (Commit commit in commits)
{
	// skip squashed commits if feature commits should be included instead
	if (queryParams.IncludeSquashedFeatureCommits && squashedCommits.Contains(commit.Sha))
	{
		continue;
	}
	processCommit(commit.CommitTime.Date, commit.Author!.ToContributorDto());
}


if (queryParams.IncludeSquashedFeatureCommits)
{
	foreach (PullRequest pr in pullRequests)
	{
		if (IsPullRequestToMainSquashed(pr, commits))
		{
			foreach (PullRequestCommitEntity commit in pr.PullRequestCommits)
			{
				processCommit(commit.CommitTime.Date, commit.Author!.ToContributorDto());;		
			}	
		}
	}
}

return commitsCountPerDate;
\end{lstlisting}

Der in \autoref{lst:graphql-commits} dargestellte Code implementiert die Zählung von Commits pro Contributor und Datum, wobei optional auch gesquashte Feature-Commits berücksichtigt werden können. Die gezeigte Methode nutzt dazu die Hilfsmethode \textit{processCommit}, welche die Zählwerte in einer verschachtelten Dictionary-Struktur (\textit{commitsCountPerDate}) aggregiert.

Zuerst werden alle Commits des Main-Branches prozessiert. Ist die Option \textit{IncludeSquashedFeatureCommits} aktiviert, so werden gesquashte Commits aus dem Main-Branch übersprungen. In einem zweiten Schritt werden alle Pull Request Commits, welche den Main- / Master-Branch als Target besitzen, in die Liste miteinbezogen. 

\section{Vorstellung der GitHub-Organisationen}
\label{sec:VorstellungGithubOrgs}
Für den Vergleich der Korrelationsanalyse der Racetrack-Repositories werden drei GitHub-Organisationen ausgewählt. Diese unterscheiden sich in ihrer Unternehmensgrösse, der Anzahl ihrer Projekte, der Grösse der jeweiligen Projekte sowie im Anteil der Repositories, die nicht nur öffentlich zugänglich sind, sondern auch aktiv als OpenSource-Projekte mit Beiträgen externer Contributor weiterentwickelt werden.

\textit{Ubique Innovation AG} ist ein Schweizer Unternehmen mit Sitz in Zürich, das sich auf die Entwicklung von Softwarelösungen mit Fokus auf Mobile-App-Entwicklung spezialisiert hat. Im Jahr 2021 beschäftigte die Firma rund 50 Mitarbeiter \cite{noauthor_mathias_2021}. Zu den bekanntesten Projekten von Ubique gehören unter anderem die Mobile App der \textbf{Schweizerischen Bundesbahnen (SBB)}, die Kartenapplikation \textbf{Swisstopo} des Bundesamt für Landestopografie sowie die \textbf{Rega App}. \parencite{noauthor_apps_nodate}.

todo: mit 8knot github org aluege (aktuell down) 
https://metrix.chaoss.io/

Die \textit{Schweizerische Bundesbahnen (SBB)} ist die staatliche Eisenbahngesellschaft der Schweiz \parencite{uvek_verkehr_energie_und_kommunikation_eidgenossisches_departement_fur_umwelt_schweizerische_nodate}. Die SBB repräsentiert eine mittelgrosse Organisation im Bereich der Open Source Entwicklung mit 104 Projekten und 137 Followers auf Github. \parencite{noauthor_swiss_nodate} 
todo: same 8knot analyse

\textit{Zalando} ist eine grosse Github-Organisation, die für ihre umfangreichen Open-Source-Projekte im Bereich der E-Commerce- und Cloud-Computing-Technologien bekannt ist. So besitzt Zalando 50 Repositories mit 835 Followern. \parencite{noauthor_zalando_nodate}


\section{Datengrundlagen}
\label{sec:Datengrundlagen}
Als Datengrundlage dienen die Repositories aus dem ersten Projekt Racetrack, welches im zweiten Projektmodul stattfindet. Dieses dauert etwa vier Wochen, wobei die genaue Anzahl der Tage pro Klasse variieren kann. Das Grundgerüst des Sourcecodes wird in der Aufgabenstellung mitgeliefert, was die Vergleichbarkeit der Projekte erhöht. Es werden 7 Teilzeit- und 5 Vollzeitklassen aus den Jahren 2021 - 2024 analysiert. Insgesamt werden 71 Racetrack Repositories untersucht. 
