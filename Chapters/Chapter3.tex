% Indicate the main file. Must go at the beginning of the file.
% !TEX root = ../main.tex

%----------------------------------------------------------------------------------------
% CHAPTER TEMPLATE
%----------------------------------------------------------------------------------------


\chapter{Vorgehen / Methoden} % Main chapter title

\label{Chapter3} % Change X to a consecutive number; for referencing this chapter elsewhere, use \ref{ChapterX}

%----------------------------------------------------------------------------------------
% SECTION 1
%----------------------------------------------------------------------------------------

\section{Mining der Repositories}
Für die ersten Experimente wurden die Repositories der Racetrack-Projekte ausgewertet. Wie im Kapitel \secref{sec:Projektmodule} erwähnt, findet dieses Projekt im Kontext des Projektmoduls 2 als erstes Projekt statt. Dieses dauert ungefähr vier Wochen, wobei die genaue Anzahl der Tage pro Klasse variieren kann. Das Grundgerüst des Sourcecodes wird in der Aufgabenstellung mitgeliefert, was die Vergleichbarkeit der Projekte erhöht. Ausserdem haben die Studenten schon Erfahrung mit der Handhabung von PullRequests. 

Damit alle Analysen durchgeführt werden konnten, musste das Mining der Projekte erweitert werden. So wurde die GraphQL Abfrage für die PullRequests erweitert, um zusätzlich den Churn als auch die Commits auf den Pull Requests zu speichern. Die Abfrage der Daten wurde durch einen neuen PullRequest Statistik API Controller ermöglicht, welcher sämtliche Daten der einzelnen PullRequests, wie etwa der Churn, die einzelnen Commits des PullRequests als auch die Anzahl geänderter Dateien zurückgibt. 

Als Datengrundlage dienten uns die Repositories mehrerer Schulklassen aus unterschiedlichen Jahrgängen. Es wurden 7 Teilzeit- als auch 5 Vollzeitklassen aus den Jahren 2021 - 2024 analysiert. Insgesamt wurden 71 Racetrack Repositories analysiert. 

Für die einzelnen Analysen wurde jeweils ein neues Jupyter Notebook erstellt. Dieser ermöglichen eine schnelle und unkomplizierte (graphische) Auswertung der Daten. \parencite{stumpf_simon_repo-detectivesba-metric-analysis-scripts_nodate}

\subsection{Outliers}
Damit keine Outliers die Datenanalyse beeinflussen, wurden diese vorgängig entfernt. Sämtliche Outliers (insegsamt 3 PRs) gehörten alle zum gleichen Projekt. Diese Pull Requests umfassen jeweils über \textit{100'000} Änderungen.

\section{Analyse Churn vs Latency}
In der Literatur finden sich unterschiedliche Aussagen zum Einfluss der Code-Grösse (Churn) auf die Dauer von Pull Requests (PR Latency). Während einige Studien einen Zusammenhang feststellen, bleiben andere Untersuchungen uneindeutig \parencite{hasan_understanding_2023}\parencite{kudrjavets_small_2022}.

Zur Analyse dieser Metriken wurde ein Jupyter Notebook \textit{latency-churn-analysis} erstellt \parencite{stumpf_simon_repo-detectivesba-metric-analysis-scripts_nodate}. 


\section{Analyse Pull Request Akzeptanz}