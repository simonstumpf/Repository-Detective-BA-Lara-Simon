% Indicate the main file. Must go at the beginning of the file.
% !TEX root = ../main.tex

%----------------------------------------------------------------------------------------
% CHAPTER TEMPLATE
%----------------------------------------------------------------------------------------


\chapter{Vorgehen / Methoden} % Main chapter title

\label{Chapter3} % Change X to a consecutive number; for referencing this chapter elsewhere, use \ref{ChapterX}

%----------------------------------------------------------------------------------------
% SECTION 1
%----------------------------------------------------------------------------------------

\section{Mining der Repositories}


Damit alle Analysen durchgeführt werden konnten, musste das Mining der Projekte erweitert werden. So wurde die GraphQL Abfrage für die PullRequests erweitert, um zusätzlich den Churn als auch die Commits auf den Pull Requests zu speichern. Die Abfrage der Daten wurde durch einen neuen PullRequest Statistik API Controller ermöglicht, welcher sämtliche Daten der einzelnen PullRequests, wie etwa der Churn, die einzelnen Commits des PullRequests als auch die Anzahl geänderter Dateien zurückgibt. 

Als Datengrundlage dienten uns die Repositories mehrerer Schulklassen aus unterschiedlichen Jahrgängen. Es wurden 7 Teilzeit- als auch 5 Vollzeitklassen aus den Jahren 2021 - 2024 analysiert. Insgesamt wurden 71 Racetrack Repositories analysiert. 

Für die einzelnen Analysen wurde jeweils ein neues Jupyter Notebook erstellt. Dieser ermöglichen eine schnelle und unkomplizierte (graphische) Auswertung der Daten. \parencite{stumpf_simon_repo-detectivesba-metric-analysis-scripts_nodate}

\subsection{Outliers}
Damit keine Outliers die Datenanalyse beeinflussen, wurden diese vorgängig entfernt. Sämtliche Outliers (insegsamt 3 PRs) gehörten alle zum gleichen Projekt. Diese Pull Requests umfassen jeweils über \textit{100'000} Änderungen.

\section{Analyse Churn vs Latency}
In der Literatur finden sich unterschiedliche Aussagen zum Einfluss der Code-Grösse (Churn) auf die Dauer von Pull Requests (PR Latency). Während einige Studien einen Zusammenhang feststellen, bleiben andere Untersuchungen uneindeutig \parencite{hasan_understanding_2023}\parencite{kudrjavets_small_2022}.

Zur Analyse dieser Metriken wurde ein Jupyter Notebook \textit{latency-churn-analysis} erstellt \parencite{stumpf_simon_repo-detectivesba-metric-analysis-scripts_nodate}. Für die Analyse wurden folgende Metriken via Mining extrahiert:
\begin{itemize}
    \item \textbf{Pull Request createTime:} Erstellungsdatum des PullRequests
    \item \textbf{Pull Request closeTime:} Schliesszeit des PullRequests.
    \item \textbf{Pull Request Additions:} Anzahl der hinzugefügten Codezeilen.
    \item \textbf{Pull Request Deletions:} Anzahl der gelöschten Codezeilen
\end{itemize}

Daraus wurden die Metriken \textit{latency} und \textit{churn} berechnet. Die \textit{latency} wird anhand folgender Formel berechnet:
\begin{equation}
latency = closeTime - createTime
\end{equation}

Die Zusammensetzung des \textit{churn} ist wie folgt:
\begin{equation}
churn = additions + deletions
\end{equation}

Mittels der Rangkorrelation nach Spearman kann überprüft werden, ob ein Zusammenhang zwischen zwei Variablen besteht. Dabei verwendet Spearman die Rangplätze der Daten und nicht deren Wert. Der daraus resultierende Korrelationskoeffizient r\textsubscript{s} kann zwischen -1 und 1 liegen, wobei dieser bei eins ein starker positiver Zusammenhang und bei -1 ein starker negativer Zusammenhang zwischen den zwei Variablen besteht. Wenn der Korrelationskoeffizient bei null liegt, besteht keine Korrelation. \parencite{noauthor_t-test_nodate}  \\
Die Formel \parencite{noauthor_t-test_nodate} stellt sich wie folgt zusammen: 
\begin{equation}
r_s = 1 - \frac{6 \sum d_i^2}{n(n^2 - 1)}
\end{equation}

\noindent\textbf{Legende:}
\begin{itemize}
  \item [$r_s$] Spearman-Rangkorrelationskoeffizient
  \item[$d_i$] Differenz der Ränge zwischen den zwei Variablen 
  \item[$n$] Anzahl der beobachteten Fälle
\end{itemize}

Des Weiteren wird eine Prüfung durchgeführt, um festzustellen, ob sich Unterschiede zwischen den Vollzeit- und Teilzeitklassen ergeben. Um zu analysieren, ob es Differenzen in der Handhabung von Pull-Requests gibt. Dafür wurden die bereits erwähnten Metriken \textit{latency} und \textit{churn} anhand deren Mittelwerte geprüft. Zudem wurde eine Analyse durchgeführt, um die Verteilung der \textit{latencies} über Teilzeit- und Vollzeitklassen zu ermitteln. Zu diesem Zweck wurden folgende Zeiträume definiert: [0min - 1min], [1min - 6min], [6min - 30min], [30min - 1h], [1h-2h], [2h-4h], [4h-8h], [8h-12h], [12h-24h], [24+].

Einige Teams haben die GitHub Flow Branching Strategie verwendet, während andere Teams die Git-Flow Strategie angewendet haben. Von der Git-Flow Strategie wurden jedoch nur die Branches Feature, Develop und Main genutzt. Um festzustellen, ob die Pull-Requests von Developer Branches in den Main-Branch die Verteilung der \textit{latencies} beeinflussen, wurde die obige Analyse zusätzlich ohne die Pull-Requests von den Developer Branches zu den Main Branches durchgeführt.

\subsection{Einfluss von Projektzeit}
Ein signifikanter Faktor, der bei Projektmodulen eine Rolle spielt, ist die fest definierte Abgabefrist. Dies unterscheidet diese Projekte von den Open-Source-Projekten, die in der Literatur häufig untersucht werden. Es soll nun untersucht werden, ob der Zeitpunkt der Erstellung eines Pull-Requests innerhalb der Projektlaufzeit einen Einfluss auf die Bearbeitung und vor allem auf die Dauer hat. Dafür wurde zusätzlich zu den oben genannten Metriken noch das Abgabedatum der einzelnen Klassen ermittelt. Da einige Dozenten ihr Feedback mittels Pull Request abgegeben haben, wurden alle Pull Requests herausgefiltert, die von einem Dozenten erstellt wurden. Dies wurde über das Attribut \textit{Author} ermittelt.

Es wurden verschiedene Analysen anhand eines Phyton Notebooks vorgenommen. Die erste Untersuchung analysierte, ob generell ein Zusammenhang über die \textit{latency} und \textit{churn} im Verlauf der Projektzeit festgestellt werden kann.  Um der in der Einleitung formulierten Hypothese nachzugehen, wurden die Pull Requests, die innerhalb von 30 Minuten und einer Minute gemerged wurden, ins Verhältnis zur Projektabgabe gesetzt und graphisch dargestellt. Des Weiteren wurden alle Pull Requests in Verbindung zur Abgabe gesetzt, um herauszufinden, wann die Pull Requests generell erstellt und bearbeitet wurden.  Darüber hinaus wurde analysiert, wie viele Pull Requests die Teams innerhalb der letzten 3 Projekttage und des letzten Projekttages noch abgearbeitet haben.


\section{Analyse Pull Request Akzeptanz}
Die aktuelle Literatur zeigt, dass sowohl soziale und prozessbezogene Faktoren als auch technische Merkmale berücksichtigt werden müssen. 

Um die Ursache des geschlossenen PRs zu gruppieren, wurden folgende Gruppen erstellt: 
\begin{itemize}
    \item \textbf{PRs ohne erkennbaren Grund}: Die PRs wurden ohne Kommentar geschlossen. Die Churn Grösse ist kleiner als 500. 
    \item \textbf{Issue / Feature durch anderen PR implementiert}: Das Feature wurde durch einen anderen PR implementiert und anschliessend dann geschlossen. 
    \item \textbf{PRs mit falschem Zielbranch}: Der Autor des PRs wählte den falschen Zielbranch aus. Der PR wurde anschliessend in einen anderen Branch gemerdet. 
    \item \textbf{Feature wird nicht mehr benötigt}: Das Feature wird nicht mehr benötigt. Diess muss so im PR vermerkt sein. 
    \item \textbf{Implementierung abgelehnt}: Die Implementierung wurde abgelehnt. 
\end{itemize}
\pagebreak
\section{Analyse Entwicklungsaktivität}
Für die Analyse der Entwicklungsaktivität wurde das Jupyter Notebook \textit{analyse-wochen-latency-churn.ipynb} erstellt \parencite{stumpf_simon_repo-detectivesba-metric-analysis-scripts_nodate}. 

Um die Entwicklungsaktivität messbar machen zu können, mussten für alle Klassen die Projektmodul-Unterrichtstage ermittelt werden:
\begin{table}[ht]
\caption{Projektmodul-Unterrichtstage der Klassen}
\label{tab:stundenplan}
\centering
\begin{tabular}{l l l}
\toprule
\textbf{Klasse} & \textbf{Ort} & \textbf{Tag} \\
\midrule
It21tb   & Zürich      & Montag      \\
It21ta   & Zürich      & Montag      \\
It21a    & Zürich      & Mittwoch    \\
It21tb   & Winterthur  & Donnerstag  \\
It21a    & Winterthur  & Freitag     \\
It21b    & Winterthur  & Freitag     \\
It21ta   & Winterthur  & Freitag     \\
\midrule
It23tb   & Zürich      & Montag      \\
It23ta   & Zürich      & Montag      \\
It23a    & Zürich      & Mittwoch    \\
It23b    & Zürich      & Mittwoch    \\
It23a    & Winterthur  & Freitag     \\
It23ta   & Winterthur  & Freitag     \\
It23tb   & Winterthur  & Freitag     \\
\midrule
It24tb   & Zürich      & Montag      \\
It24ta   & Zürich      & Montag      \\
It24a    & Zürich      & Mittwoch    \\
It24a    & Winterthur  & Freitag     \\
It24ta   & Winterthur  & Freitag     \\
It24tb   & Winterthur  & Freitag     \\
\bottomrule
\end{tabular}
\end{table}

