% Indicate the main file. Must go at the beginning of the file.
% !TEX root = ../main.tex

%----------------------------------------------------------------------------------------
% EINLEITUNG
%----------------------------------------------------------------------------------------


\chapter{Einleitung} % Main chapter title


\label{Chapter1} % Change X to a consecutive number; for referencing this chapter elsewhere, use \ref{ChapterX}

%----------------------------------------------------------------------------------------
% SECTION 1
%----------------------------------------------------------------------------------------

\section{Ausgangslage}
\label{sec:Ausgangslage} 
Code Reviews sind ein wichtiger Bestandteil der modernen Softwareentwicklung. Reviews fördern nicht nur die Qualität des Codes, sondern auch die Zusammenarbeit im Team \parencite{dos_santos_investigating_2018}.  In der Praxis werden Code Reviews häufig über Plattformen wie GitHub durchgeführt. Dabei werden auf der Plattform sogenannte Pull Requests erstellt, welche dann als Grundlage für die Reviews dienen.

Die Analyse von Pull Requests kann wertvolle Informationen über die Zusammenarbeit und den Fortschritt innerhalb eines Softwareprojekts liefern. Eine automatisierte Auswertung von Pull Request Daten bietet die Möglichkeit, Probleme oder Verbesserungsmöglichkeiten zu erkennen. Diese können sowohl auf technischer als auch sozialer Ebene sein. 

An der Zürcher Hochschule für Angewandte Wissenschaften (ZHAW) lernen die Studierenden in Projektmodulen, wie man Projekte erfolgreich durchführt. Für die Zusammenarbeit der Teams wird GitHub eingesetzt. Eine automatisierte Auswertung der Pull Requests kann den Dozierenden helfen, Probleme bei den Studierenden frühzeitig zu erkennen.


%----------------------------------------------------------------------------------------
% SECTION 2
%----------------------------------------------------------------------------------------

\section{Zielsetzung / Aufgabenstellung / Anforderungen}

Ziel dieser Arbeit ist es, Pull Requests aus Projekten zu untersuchen und deren Eigenschaften hinsichtlich verschiedener Einflussfaktoren zu analysieren. Dabei stehen folgende Forschungsfragen im Vordergrund:

\textbf{Forschungsfrage 1}:
Besteht ein Zusammenhang zwischen der Latency eines Pull Requests (Zeit bis zum Merge oder Schliessung) und dem Churn (Anzahl geänderter Codezeilen) des jeweiligen Pull Requests?

\textbf{Forschungsfrage 2}:
Lassen sich die geschlossenen Pull Requests aus den Projektmodul-Projekten in sinnvolle Kategorien einteilen?

\textbf{Forschungsfrage 3}:
Lassen sich Patterns erkennen an welchen Tagen die Studierenden an den Projekten arbeiten.
\newpage
Des Weiteren sollen die folgenden Hypothesen anhand von Projekten der Studierenden untersucht werden.

\textbf{Hypothese 1:
Gegen Ende des Projekts werden Pull Requests schneller gemergt.}\\
Die Projekte haben ein festes Ziel und einen festen Abgabetermin. Aus eigener Erfahrung der Autoren schieben Studierende gerne Arbeiten vor sich her und generieren dadurch mehr Arbeit für das Ende des Projektes. Aus diesem Grund gehen wir davon aus, dass Pull Requests gegen Ende des Projekts schneller bearbeitet werden.

\textbf{Hypothese 2:
Teilzeitstudierende nutzen Pull Requests effizienter als Vollzeitstudierende.}\\
Wir gehen davon aus, dass Teilzeitstudierende durch ihre Arbeit mehr Erfahrung in der Durchführung von Softwareprojekten und somit auch in der Verwendung von Pull Requests haben. Dies kann sich in Anzahl, Grösse aber auch Kommentare unterscheiden. 

Die vorliegende Arbeit wurde für ein Zielpublikum verfasst, das in der Informatikbranche tätig ist.


\section{Stand der Technik} % Main chapter title

\label{Chapter3} % Change X to a consecutive number; for referencing this chapter elsewhere, use \ref{ChapterX}

%----------------------------------------------------------------------------------------
% SECTION 1
%----------------------------------------------------------------------------------------
In diesem Kapitel wird der gegenwärtige Stand der Technik erörtert.
\subsection{Pull Request Dauer}
Es existieren bereits diverse Studien, die sich mit der Analyse von Pull Requests befassen. In der Literatur werden verschiedene Faktoren untersucht, die die Dauer von Pull Requests beeinflussen. Zusammenfassend kommen die Studien zum Schluss, dass die Pull Request Laufzeit von verschiedenen Faktoren, wie der Diskussion, Reputation des Erstellers, Team-Auslastung, erster menschlicher Reaktion und Continuous Integration abhängig sind. Es existieren jedoch widersprüchliche Aussagen bezüglich der Abhängigkeit der Laufzeit von der PR-Grösse. \parencite{yu_wait_2015}\parencite{hasan_understanding_2023}\parencite{kudrjavets_small_2022}\parencite{bernardo_studying_2018}

Ein Beispiel ist das Paper von Yue Yu u.a.\parencite{yu_wait_2015}, welches  mittels linearer Regressionsmodelle relevante Faktoren bei GitHub-Projekten analysiert hat. Die Resultate der Untersuchung zeigen, dass Faktoren wie die Grösse – definiert als die Anzahl der Änderungen des Pull Requests – oder die Länge der Diskussion über den Pull Request die Laufzeit beeinflussen. Des Weiteren wurde ein Zusammenhang zwischen der Reputation des Stellers und der Dauer nachgewiesen. Darüber hinaus wurde ein Zusammenhang zwischen der Grösse des Teams, der Auslastung - definiert als Anzahl offener Pull Requests - sowie der Zeit bis zur ersten menschlichen Reaktion auf den Pull Request festgestellt. Zusätzlich wurde gezeigt, dass CI-Faktoren einen signifikanten Einfluss haben, wie beispielsweise das Warten auf automatisierte Tests.~\parencite{yu_wait_2015}

Das Paper von Kazi Amit Hasan u.a.\parencite{hasan_understanding_2023} befasst sich spezifischer mit der ersten Reaktion auf einen Pull Request anhand von Open Source Projekten. In diesem Zusammenhang sind Reaktionen definiert in Form eines Kommentars auf einen Pull Request oder eines Code-Review-Kommentars auf einen Pull Request. Es wurde festgestellt, dass die erste Antwort eines Bots kaum Auswirkungen auf die Dauer hat, die erste Antwort eines Menschen jedoch einen grossen Einfluss hat. Die Dauer der ersten menschlichen Reaktion ist in der Hälfte der Fälle auf neun Faktoren zurückzuführen. Diese neuen Faktoren sind die Länge der Beschreibung, die Anzahl Commits, die Anzahl bearbeiteter Files, die Anzahl geänderter Codezeilen, die Anzahl hinzugefügter Codezeilen,  ob eine Person auf dem Pull Request markiert wurde, die Anzahl offener Pull Requests, die Anzahl der bisherigen Pull Requests des Entwicklers und der Anteil der Teammitglieder, mit denen der Entwickler in den letzten drei Monaten interagiert hat. Insbesondere Pull-Requests, die beim Eröffnen eine längere Beschreibung und komplexere Code-Änderungen haben, dauert die erste menschliche Reaktion länger. Ebenfalls stellten sie einen Zusammenhang bezüglich der Auslastung, also den offenen PRs und der ersten Interaktion fest. Ein weiteres Resultat ihrer Recherche zeigte auf, dass Verfasser mit weniger Erfahrung (Anzahl erfasster PRs) und fehlende Interaktionen mit Projektmitgliedern ebenfalls zu längeren ersten menschlichen Reaktionen führen kann.\parencite{hasan_understanding_2023} 

Beide oben beschriebenen Studien stellen einen Zusammenhang zwischen der Grösse eines Pull Requests und dessen Dauer fest \parencite{yu_wait_2015}\parencite{hasan_understanding_2023}. Die Studie von Gunnar Kudrjavets, Nachiappan Nagappan und Ayushi Rastogi  \parencite{kudrjavets_small_2022}, die sich spezifisch mit dieser Thematik befasst, gelangt zu dem Schluss, dass kein Zusammenhang zwischen diesen beiden Aspekten festgestellt werden kann.

Das Paper von João Helis Bernardo, Daniel Alencar Da Costa und Uirá Kulesza \parencite{bernardo_studying_2018} fand heraus, dass Projekte, die Continuous Integration eingeführt haben, danach Pull Requests mit längeren Laufzeiten hatten.




%----------------------------------------------------------------------------------------
% SECTION 2
%----------------------------------------------------------------------------------------

\subsection{Pull Request Akzeptanz}
Die Akzeptanz eines Pull Requests bestimmt, ob ein Pull Request gemerged oder abgelehnt (rejected) wird. Über die Einflussfaktoren der Akzeptanz von Pull Requests existieren bereits mehrere Studien.
So zeigen diverse Untersuchungen, dass die Akzeptanz eines Pull Requests von mehreren technischen als auch sozialen Faktoren abhängig ist \parencite{gousios_exploratory_2014}. \
Eine Studie, die über 40'000 Pull Requests in Projekten mit unterschiedlichen Programmiersprachen (Java, Python, C etc.) analysierte, stellte fest, dass die Code-Qualität nur einen geringen Einfluss auf die Akzeptanzrate hat \parencite{kuhejda_pull_2023}.

Die Grösse eines Pull Requests (Umfang der Code-Änderungen) ist ein technisches Merkmal, das in der Literatur häufig erwähnt wird. Hierzu gibt es in der Forschung unterschiedliche Ergebnisse. So stellte eine Studie, die manuell über 300 Pull Requests der Eclipse- und Mozilla-Foundation untersuchte, fest, dass nur 0.3\,\% aller abgelehnten Pull Requests aufgrund ihrer zu grossen Grösse abgelehnt wurden \parencite{tao_writing_2014}. Andere Untersuchungen ergaben, dass umfangreiche Pull Requests tendenziell seltener akzeptiert werden. Ein statistisches Modell zeigte, dass mit jeder zusätzlichen Grössenordnung die Wahrscheinlichkeit einer Akzeptanz um 26\,\% sinkt. Allerdings wird betont, dass die Grösse eines Pull Requests allein selten der Ablehnungsgrund ist. Vielmehr erschweren grosse PRs die Bewertung und führen dann zu einer Ablehnung, wenn zusätzliche Kosten, Unklarheiten oder Qualitätsmängel auftreten könnten. \ Eine Studie, die sich auf Mozilla-Projekte konzentrierte, stellte fest, dass Entwickler häufig glauben, dass die Grösse eines Pull Requests entscheidend für die Akzeptanz ist. \parencite{tsay_influence_2014}

Darüber hinaus haben auch zeitliche und prozessuale Faktoren einen grossen Einfluss auf die PR-Akzeptanz. Eine Untersuchung der zeitlichen Dimension von PR-Entscheidungen zeigt, dass sich die Bedeutung einzelner Variablen im Verlauf eines PR-Lebenszyklus ändern kann. Die wichtigsten Variablen für die finale Merge-Entscheidung (bei bereits abgeschlossenen PRs) sind zwar auch für offene Pull Requests relevant, können jedoch gegenteilig wirken. Das bedeutet, dass bestimmte Merkmale wie etwa die Grösse, Tests oder die Diskussion über den PR in früheren Phasen anders bewertet werden als im Endergebnis. \parencite{west_temporal_2023}

Soziale Faktoren spielen ebenfalls eine wichtige Rolle. Pull Requests von Entwicklern, die eine soziale Bindung zum Projekt haben (durch vorherige Mitarbeit oder durch den persönlichen Kontakt mit einem Projekt-Maintainer), werden deutlich häufiger akzeptiert. Ebenso begünstigt das vorherige Mitwirken im Projekt, etwa durch frühere PRs oder Issues, die Annahme eines Pull Requests. Umfangreiche Diskussionen reduzieren jedoch wiederum die Wahrscheinlichkeit einer Akzeptanz \parencite{tsay_influence_2014}.

Insgesamt zeigen die Studien, dass technische Merkmale (Code-Qualität, Änderungsumfang, Tests etc.) stets im Zusammenspiel mit sozialen und prozessbezogenen Faktoren betrachtet werden müssen, um die Akzeptanz von Pull Requests zu erklären.